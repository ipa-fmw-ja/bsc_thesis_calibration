\label{dh-p}

Die Denavit-Hartenberg Notation basiert auf einem von Denavit und Hartenberg 
entwickelten Verfahren zur Berechnung von Transformationen entlang einer 
kinematischen Kette. Dazu werden für jedes 
Gelenk vier Parameter benötigt. Die sogenannten \ac{DH-Parameter}. Aus diesen
Parametern kann dann für jedes Gelenk die Transformationsmatrix zum vorherigen 
Gelenk berechnet werden. Das Verfahren vereinfacht vor allem die Berechnung 
der Vorwärtskinematik, also die Berechnung von Positionen anhand der 
Gelenkzustände. 

Um eine Transformation mit vier statt den sonst üblichen 
sechs Parametern zu beschreiben, werden die möglichen Freiheitsgrade der Gelenke
limitiert. Um einen Aktor mit \ac{DH-Parameter}n zu beschreiben dürfen alle
Gelenke nur einen Freiheitsgrad haben. In der Anwendung führt dies aufgrund der
gebräuchlichen Roboterbauformen - Schubgelenk oder Drehgelenk - aber selten zu Einschränkungen.

Zur Angabe einer Transformation werden die 
Parameter $[a, \alpha, d, \Theta]$ angegeben. Daraus kann die homogene 
Translationsmatrix berechnet werden.\cite{craig2005}
