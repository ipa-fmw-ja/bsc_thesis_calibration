\label{dh-p}

Denavit und Hartenberg entwickelten 1955 ein Verfahren zur Berechnung von 
Transformationen entlang einer kinematischen Kette. Dazu werden für jedes 
Gelenk vier Parameter benötigt. Die sogenannten \ac{DH-Parameter}. Aus diesen
Parametern kann dann für jedes Gelenk die Transformationsmatrix zum vorherigen 
Gelenk berechnet werden. Das Verfahren vereinfacht vor allem die berechnung 
der Vorwärtskinematik, also die Berechnung von Positionen anhand der 
Gelenkzustände. Um eine Transformation mit vier statt den sonst üblichen 
sechs Parametern zu beschreiben werden die möglichen Freiheitsgrade der Gelenke
limitiert. In der Anwendung führt dies aufgrund der gebräuchlichen Roboterbauformen
aber selten zu Einschränkungen.
