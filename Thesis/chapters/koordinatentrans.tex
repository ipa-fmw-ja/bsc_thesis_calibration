\section{Koordinatentransformationen}

Ein wichtiges Thema bei der Robotik ist die Transformation eines 
Koordinatensystems in ein anderes. In der Robotik bekommt jedes
Objekt, das maßgeblich an einer Operation beteiligt ist, ein eigenes
Koordinatensystem, um seine Lage, bestehend aus Position und Orientierung
im Raum zu beschreiben. Dadurch lässt sich berechnen, wie ein Objekt gegriffen
werden muss oder Kollisionen vermieden werden. Die Transformation von einem 
Koordinatensystem in ein anderes, wie sie zum Beispiel vorkommt, wenn die 
Kameras (Koordinatensystem 1) ein Objekt (Koordinatensystem 2) erkennen, kann
durch eine Translation und eine Rotation dargestellt werden. Wenn die Lage der
Kameras relativ zur Basis des Roboters (Koordinatensystem 0) bekannt ist, ist
auch die Lage des Objekts im Koordinatensystem 0 bekannt. Für den \cob\ ist 
jedes Koordinatensystem ein Kind eines sogenannten Parent-Frames, das wiederum 
ein Koordinatensystem ist. Ein Parent-Frame kann mehrere Kinder haben, aber ein
Kind kann nur einen Parent-Frame haben. Außerdem sind alle Koordinatensysteme
des \cob\ orthonormale Koordinatensysteme. 

\subsection{Translation}
\label{sub:Translation}


Die Translation beschreibt nur die Position eines Punktes im Raum. Für die 
Koordinatentransformation ist dieser Punkt der Ursprung des neuen Koordinatensystems.
Der Punkt P wird im n-Dimensionalen durch einen Koordinatenvektor mit n 
Elementen angegeben. Räumliche Transformationen beschränken sich auf drei Dimensionen
und damit den Punkt $P= [p_x\\p_y\\p_z]^T$. Jedes Element $p_n$ gibt die Entfernung
entlang der n-ten Achse des Parent-Frames an.

Da für jeden Punkt festgelegt sein muss, in welchem Koordinatensystem $A$ er angegeben
ist, wird das Bezugskoordinatensystem in der Form $^AP$ angegeben.

\subsection{Rotation}
\label{sub:Rotation}

Da in der Robotik die Position eines Objektes selten ausreichend ist, wird zusätzlich 
die Orientierung benötigt. Dafür bekommt das Objekt ein Koordinatensystem bestehend
aus dem Ursprung $P$ und drei Achsen. Die Ausrichtung der drei Achsen werden durch
drei Vektoren im Parent-Frame angegeben, die orthogonal sind und die Länge $1$
haben. Die drei Vektoren, die die Achsen des Koordinatensystems $B$ im Parent-Frame
$A$ angeben, werden eindeutig mit $^A\hat{X}_B$, $^A\hat{Y}_B$ und $^A\hat{Z}_B$
angegeben. Diese drei Vektoren werden zu der sogenannten Rotationsmatrix
\begin{equation}
  ^A_BR = [ ^A\hat{X}_B ^A\hat{Y}_B ^A\hat{Z}_B ] = \begin{pmatrix}
    r_{11}&r_{12}&r_{13}\\
    r_{21}&r_{22}&r_{23}\\
    r_{31}&r_{32}&r_{33}
  \end{pmatrix}
  \label{eq:rotationsmatrix}
\end{equation}
zusammengefasst.

Für die Rotation $^B_AR$ gilt wie in \cite{craig2005} angegeben:
\begin{equation}
  ^B_AR=^A_BR^T
\end{equation}


Zur vereinfachten Darstellung von Rotationen genügen drei Winkel, aus denen die
Rotationsmatrix berechnet werden kann. Am \cob\ wird hierfür die \begin{quote}Roll-Pitch-Yaw\end{quote}
Präsentation gewählt. Weitere verbreitete Methoden, sind wie in \cite{sciavicco2000modelling} 
Dargestellt die zwölf Versionen der Euler-Winkel.

\subsubsection{Roll-Pitch-Yaw}
\label{ssub:Roll-Pitch-Yaw}

Die Roll-Pitch-Yaw Beschreibung stammt aus der Luft- und Seefahrt und wird genutzt
um die Orientierung eines Objekts mit den Winkeln Rollwinkel, Gierwinkel und 
Nickwinkel anzugeben.
\begin{description}
  \item[$\phi$]Rollwinkel: Winkel, der eine Drehung um die X-Achse des Referenzkoordinatensystems
    angibt
  \item[$\psi$]Gierwinkel: Winkel, der die Drehung um die Z-Achse eines Referenzkoordinatensystems
    beschreibt
  \item[$\theta$]Nickwinkel: Winkel, der die Rotation um die feste Y-Achse beschreibt.

\end{description}


Die Rotationsmatrix berechnet sich daraus mit $c_x=cos(x)$ und $s_x=sin(x)$ zu:

\begin{equation}
  R_{RPY}=
  \begin{pmatrix}
    c_\phi*c_\theta     
    & c_\phi*s_\psi*s_\theta-s_\phi*c_\theta
    & c_\phi*s_\psi*c_\theta-s_\phi*s_\theta\\

    s_\phi*s_\psi          
    & s_\phi*s_\psi*s_\theta+c_\phi*c_\theta
    & s_\phi*s_\psi*c_\theta-c_\phi*s_\theta\\
  
    -s_\psi    
    & c_\psi*s_\theta   
    & c_\phi*c_\theta
  
  \end{pmatrix}
\end{equation}

Alle für den \cob\ angegebenen Orientierungen in der Roboterbeschreibung sind 
als Roll-Pitch-Yaw Winkel angegeben. Außerdem kann durch vorhandene Nodes 
die Orientierung von jedem Koordinatensystem des \cob\ relativ zu jedem 
anderen Koordinatensystem des \cob\ in einem Translationsvektor und den Roll-Pitch-Yaw
Winkeln berechnet werden.


\subsubsection{Denavit-Hartenberg Parameter}
\label{ssub:Denavit-Hartenberg Parameter}
\label{dh-p}

Die Denavit-Hartenberg Notation basiert auf einem von Denavit und Hartenberg 
entwickelten Verfahren zur Berechnung von Transformationen entlang einer 
kinematischen Kette. Dazu werden für jedes 
Gelenk vier Parameter benötigt. Die sogenannten \ac{DH-Parameter}. Aus diesen
Parametern kann dann für jedes Gelenk die Transformationsmatrix zum vorherigen 
Gelenk berechnet werden. Das Verfahren vereinfacht vor allem die Berechnung 
der Vorwärtskinematik, also die Berechnung von Positionen anhand der 
Gelenkzustände. 

Um eine Transformation mit vier statt den sonst üblichen 
sechs Parametern zu beschreiben, werden die möglichen Freiheitsgrade der Gelenke
limitiert. Um einen Aktor mit \ac{DH-Parameter}n zu beschreiben dürfen alle
Gelenke nur einen Freiheitsgrad haben. In der Anwendung führt dies aufgrund der gebräuchlichen Roboterbauformen
mit entweder Schubgelenken oder Drehgelenken aber selten zu Einschränkungen.

Zur Angabe einer Transformation werden die 
Parameter $[a, \alpha, d, \Theta]$ angegeben. Daraus kann die homogene 
Translationsmatrix berechnet werden.\cite{craig2005}

