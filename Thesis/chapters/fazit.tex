\chapter{Zusammenfassung und Ausblick}

Das Ziel dieser Arbeit war es, die automatische Kalibrierung des \cob\ anzupassen 
und zu generalisieren. Hierdurch sollte ein einheitliches Kalibrierverfahren 
entwickelt werden, mit dem alle \cob\ und andere am \ac{IPA} eingesetzte 
Serviceroboter kalibriert werden können. Die bereits vorhandene Lösung erzielt für
den \cob3-3 gute Ergebnisse, mit denen eine sichere Objekterkennung und
Manipulation möglich ist.
Diese Lösung konnte allerdings nicht auf andere Roboter angewandt werden.

Zur Generalisierung wurde das bisher am \cob3-3 eingesetzte Verfahren analysiert
und das Verbesserungspotential festgestellt.

Die beiden Datenaufnahmeschritte zur Kamera- und kinematischen Kalibrierung  
des alten Verfahrens wurden zu einem Schritt zusammengefasst.
Dafür werden zur 
kinematischen Kalibrierung die Merkmale eines Kalibrierobjekts auf den
verzerrten Bildern der Kameras erkannt. Bei der Berechnung wird diese
Verzerrung im Modell berücksichtigt.
Dadurch ist die Aufnahme eines Datensatzes zur Kalibrierung mit unkalibrierten
Kameras möglich und die Datenaufnahme zur Kamerakalibrierung entfällt. 

Zur Berechnung der Kalibrierung wurden im Kalibrierverfahren \ac{DH-Parameter}
benötigt, die nicht für alle behandelten Roboter verfügbar waren. Dadurch 
konnte die Kalibrierung nur auf einem Roboter ausgeführt werden.
Um Roboter ohne verfügbare \ac{DH-Parameter} zu kalibrieren, wurde die Berechnung
der Vorwärtskinematik dahingehend angepasst, dass das vorhandene Modell des 
Roboters dazu eingesetzt wird. Dadurch entfällt zusätzlich die manuelle Eingabe der 
\ac{DH-Parameter}.

Die Implementierung erfolgt im \ac{ROS}-Stack \texttt{cob\_calibration}.
Um neue Roboter einfacher in die Kalibrierung aufzunehmen, wurden alle
Konfigurationen im neuen \ac{ROS}-Package \texttt{cob\_calibration\_config}
zusammengefasst.
Anstatt festgelegter Positionen zur Datenaufnahme werden Samplepositionen für
die Roboter berechnet.  
Dadurch können Roboter mit unterschiedlichen Aktoren in die Kalibrierung einbezogen 
werden.


Durch diese Änderungen wird das Hauptziel der Bachelorthesis, das
Kalibrierungsverfahren für die Kameras und die beiden "`relevanten 
Komponenten"'\cite{Haug2012} Arm und Torso anzupassen, um andere Roboter
zu kalibrieren, erreicht.

Zusätzlich wurde ein Verfahren entwickelt, mit dem ein Kalibrierobjekt in den 
2D Laserscan Daten erkannt werden kann. Zur Kalibrierung der Laserscanner zu 
den Kameras kann das Verfahren in zukünftigen Arbeiten verbessert und stabilisiert
werden. Außerdem muss die mathematische Optimierung implementiert werden.

Außerdem können in der Zukunft weitere Komponenten des \cob\ zum Beispiel die Montageposition 
des Tabletts durch taktile Messungen oder angebrachte Marker kalibriert werden.

