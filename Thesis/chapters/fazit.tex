\chapter{Ausblick und Fazit}

Das Ziel dieser Arbeit ist die automatische Kalibrierung des \cob\ anzupassen 
und zu generalisieren. Hierdurch soll ein einheitliches Kalibrierverfahren 
geschaffen werden mit dem alle \cob\ und andere am \ac{IPA} eingesetzte 
Serviceroboter kalibriert werden können. Die bereits vorhandene Lösung führt am
\cob3-6 zu guten Ergebnissen, mit denen eine sichere Objekterkennung und
Manipulation möglich ist. Diese Lösung kann nicht auf andere Roboter angewandt
werden. Durch die genaue Untersuchung des \cob\ sowie des bisherigen
Kalibrierablauf werden die Softwarekomponenten erkannt, die eine Kalbrierung 
verhindern. 

Die beiden Kalibrierungsstufen zur Kamera- und kinematischen Kalibrierung  des
alten Verfahrens werden durch die Änderungen zusammengefasst. Dafür werden zur 
kinematischen Kalibrierung die Merkmale eines Kalibrierobjekts auf den
verzerrten Bildern der Kameras erkannt. Bei der berechnung wird diese
Verzerrung berücksichtigt. Dadurch ist die Aufnahme eines Datensatzes zur
Kalibrierung mit unkalibrierten Kameras möglich und die Datenaufnahme zur
Kamerakalibrierung entfällt. 

Zur Berechnung der Kalibrierung werden im bisherigen Verfahren \ac{DH-Parameter}
benötigt, die nicht für alle behandelten Roboter verfügbar sind. Dadurch 
kann die Kalibrierung nur auf einem Roboter ausgeführt werden. Zur Generalisierung
werden diese durch das vorhandene Modell des Roboters ersetzt. Durch diese 
Anpassung können auch Roboter mit Aktoren für die keine \ac{DH-Parameter} 
verfügbar sind kalibriert werden.

Die Implementierung erfolgt als \ac{ROS}-Stack. Um neue Roboter einfacher in 
die Kalibrierung aufzunehmen werden alle Konfigurationen im neuen
\ac{ROS}-Package \texttt{cob\_calibration\_config} zusammengefasst. Die
Bestimmung von neuen Samplepositionen erfolgt anstatt festgelegter Positionen 
durch eine Strategie die gültige Samples für die Roboter berechnet. Dadurch 
können Roboter mit unterschiedlichen Aktoren in die Kalibrierung einbezogen 
werden. 

Das Hauptziel der Bachelorthesis das Kalibrierungsverfahren für die Kameras und
die beiden ``relevanten Komponenten'' Arm und Torso anzupassen um andere Roboter
zu kalibrieren wird durch diese Änderungen erreicht. Durch den Vergleich der
Messdaten der Kamera und des Modells kann die Genauigkeit der Kalibrierung 
überprüft werden. Außerdem ist nach der Kalibrierung das erfolgreiche Greifen 
von erkannten Gegenständen möglich. 

In weiteren Arbeiten kann die Kalibrierung noch um die Montageposition der
Laserscanner oder des Tabletts erweitert werden. Am \cob\ werden nur 2D-Laserscanner
eingesetzt die keine Helligkeitswerte zurückgeben. Zur Kalibrierung kann wie 
bisher ein Objekt erkannt und ein Modell an die Messwerte angepasst werden.
Die Montageposition des Tabletts kann zum Beispiel durch taktile Messungen oder
Kalibriermuster am Tablett berechnet werden.

