%
% Diplomarbeit mit LaTeX
% ===========================================================================
% This is part of the book "Diplomarbeit mit LaTeX".
% Copyright (c) 2002, 2003, 2005, 2007, 2008 Tobias Erbsland
% Copyright (c) 2005, 2006 Andreas Nitsch
% See the file main.tex for copying conditions.
%

%
% A. DOKUMENTKLASSE
% ---------------------------------------------------------------------------
%

%
%  1. Definieren der Dokumentklasse.
%     Wir verwenden die KOMA-Script Klasse 'scrbook' für ein Buch.
%
\documentclass[%
	%pdftex,%              PDFTex verwenden da wir ausschliesslich ein PDF erzeugen.
	a4paper,%             Wir verwenden A4 Papier.
	oneside,%             Einseitiger Druck.
	12pt,%                Grosse Schrift, besser geeignet für A4.
	parskip=half,%         Halbe Zeile Abstand zwischen Absätzen.
	%chapterprefix,%       Kapitel mit 'Kapitel' anschreiben.
	headsepline,%         Linie nach Kopfzeile.
	footsepline,%         Linie vor Fusszeile.
	bibliography=totoc,%    Literaturverzeichnis im Inhaltsverzeichnis nummeriert einfügen.
	index=totoc%             Index ins Inhaltsverzeichnis einfügen.
]{scrbook}
\linespread{1.5}
\usepackage[left=5cm, right=1.5cm, top=2.5cm, bottom=1cm,includeheadfoot]{geometry}

%
%  2. Festlegen der Zeichencodierung des Dokuments und des Zeichensatzes.
%     Wir verwenden 'Latin1' (ISO-8859-1) für das Dokument,
%     und die 'T1' codierung für die Schrift.

\usepackage[utf8]{inputenc}
\usepackage[T1]{fontenc}

%
%  3. Packet für die Index-Erstellung laden.
%
\usepackage{makeidx}

%
%  4. Paket für die Lokalisierung ins Deutsche laden.
%     Wir verwenden neue deutsche Rechtschreibung mit 'ngerman'.
%
\usepackage[ngerman]{babel}


%
%  5. Paket für Anführungszeichen laden.
%     Wir setzen den Stil auf 'swiss', und verwenden so die Schweizer Anführungszeichen.
%
\usepackage[babel, german=quotes]{csquotes}


%
%  6. Paket für erweiterte Tabelleneigenschaften.
%
\usepackage{array}

%
%  7. Paket um Grafiken im Dokument einbetten zu können.
%     Im PDF sind GIF, PNG, und PDF Grafiken möglich.
%
\usepackage{graphicx}

%
%  8. Pakete für mathematischen Textsatz.
%
%\usepackage{amsmath}
%\usepackage{amssymb}
%\usepackage{dsfont}
\usepackage{mathtools}
\mathtoolsset{showonlyrefs}

%
%  9. Paket um Textteile drehen zu können.
%
\usepackage{rotating}

%
% 10. Paket für Farben an verschieden Stellen. 
%     Wir definieren zusätzliche benannte Farben.
%
\usepackage{color}
\usepackage{xcolor}

%
% 11. Paket für spezielle PDF features.
%
\usepackage[%
	pdftitle={Bachelorthesis zur automatischen Kalibrierung eines mobilen Serviceroboters},%                        Titel des PDF Dokuments.
	pdfauthor={Jannik Abbenseth},%              Autor des PDF Dokuments.
	pdfsubject={Bachelorthesis},%                    Thema des PDF Dokuments.
	pdfcreator={MiKTeX, LaTeX with hyperref and KOMA-Script},% Erzeuger des PDF Dokuments.
	pdfkeywords={Kalibrierung, ROS, Robot Operating System},%                        auch für PDF Dokumente indexiert.)
	pdfpagemode=UseOutlines,%                                  Inhaltsverzeichnis anzeigen beim Öffnen
	pdfdisplaydoctitle=true,%                                  Dokumenttitel statt Dateiname anzeigen.
	pdflang=de%                                               Sprache des Dokuments.
]{hyperref}

%
% 12. Paket um Quellcode sauber zu formatieren.
%     Mit der option 'savemem' verschieben wir das laden von 
%     einzelnen Programmiersprachen auf einen späteren Zeitpunkt.
%
\usepackage[savemem]{listings}

%
% 13a. Privates Paket für die Schriftart 'Goudy Sans' laden.
%      Dieses Paket ist nur für die publizierte Version des Dokuments gedacht
%      und an dieser Stelle mit den nachfolgenden Anweisungen auskommentiert.
%
%\usepackage{goudysans}

%
% 13a. Font 'Latin Modern Family' verwenden.
%      Verwende dieses Paket wenn du DML selbst kompilierst.
%
\usepackage{lmodern}

%
% 14. Typewriter Font LuxiMono laden.
%
%\usepackage[scaled=.85]{luximono}


% 
% B. EINSTELLUNGEN
% ---------------------------------------------------------------------------
%

%
%  1. Definieren von eigenen benannten Farben.
%     Für spätere Verwendung in dem Dokument, definieren wir einzelne
%     benannte Farben.
%
\definecolor{LinkColor}{rgb}{0,0,0}
%\definecolor{LinkColor}{rgb}{0,0,0.5}
\definecolor{ListingBackground}{rgb}{0.85,0.85,0.85}

%
%  2. KOMA-Script Option, Zeilenumbruch bei Bildbeschreibungen.
%
\setcapindent{1em}

%
%  3. Stil der Kopf- und Fusszeilen.
%     Wir aktivieren mit 'headings' laufende Seitentitel.
%
\pagestyle{headings}

%
%  4. Stil der Überschriften auf normale Schrift.
%     Wir verwenden für die Überschriften den selben Font wie für den Text.
%
\setkomafont{sectioning}{\normalfont\bfseries}       % Titel mit Normalschrift
\setkomafont{captionlabel}{\normalfont\bfseries}     % Fette Beschriftungen 
\setkomafont{pagehead}{\normalfont\itshape}          % Kursive Seitentitel
\setkomafont{descriptionlabel}{\normalfont\bfseries} % Fette Beschreibungstitel

%
%  5. Farbeinstellungen für die Links im PDF Dokument.
%
\hypersetup{%
	colorlinks=true,%        Aktivieren von farbigen Links im Dokument (keine Rahmen)
	linkcolor=LinkColor,%    Farbe festlegen.
	citecolor=LinkColor,%    Farbe festlegen.
	filecolor=LinkColor,%    Farbe festlegen.
	menucolor=LinkColor,%    Farbe festlegen.
	urlcolor=LinkColor,%     Farbe von URL's im Dokument.
	bookmarksnumbered=true%  Überschriftsnummerierung im PDF Inhalt anzeigen.
}

%
%  6. Einstellungen für das 'listings' Paket.
%
\lstloadlanguages{TeX} % TeX sprache laden, notwendig wegen option 'savemem'
\lstset{%
	language=[LaTeX]TeX,     % Sprache des Quellcodes ist TeX
	numbers=left,            % Zelennummern links
	stepnumber=1,            % Jede Zeile nummerieren.
	numbersep=5pt,           % 5pt Abstand zum Quellcode
	numberstyle=\tiny,       % Zeichengrösse 'tiny' für die Nummern.
	breaklines=true,         % Zeilen umbrechen wenn notwendig.
	breakautoindent=true,    % Nach dem Zeilenumbruch Zeile einrücken.
	postbreak=\space,        % Bei Leerzeichen umbrechen.
	tabsize=2,               % Tabulatorgrösse 2
	basicstyle=\ttfamily\footnotesize, % Nichtproportionale Schrift, klein für den Quellcode
	showspaces=false,        % Leerzeichen nicht anzeigen.
	showstringspaces=false,  % Leerzeichen auch in Strings ('') nicht anzeigen.
	extendedchars=true,      % Alle Zeichen vom Latin1 Zeichensatz anzeigen.
	backgroundcolor=\color{ListingBackground}} % Hintergrundfarbe des Quellcodes setzen.

%
% C. NEUE MAKROS UND UMGEBUNGEN
% ---------------------------------------------------------------------------


%
%  1. Umgebung für Änerungsliste mit einem speziellen Aufzählungszeichen.
%
\newenvironment{ListChanges}%
	{\begin{list}{$\diamondsuit$}{}}%
	{\end{list}}

%
%  2. Ersatz für die \LaTeX und \TeX Befehle für korrekte Darstellung.
%     Wir verwenden die 'Latin Modern Family' ('lm') als Font, da diese im
%     vergleich zu 'Computer Modern' ('cm') auch PostScript Dateien
%     anbieten, was zu einer schöneren Darstellung im PDF führt.
%
\newcommand{\DMLLaTeX}{{\fontfamily{lmr}\selectfont\LaTeX}}
\newcommand{\DMLTeX}{{\fontfamily{lmr}\selectfont\TeX}}

\def\AmS{$\mathcal{A}$\kern-.1667em\lower.5ex\hbox
    {$\mathcal{M}$}\kern-.125em$\mathcal{S}$}
\def\AmSmath{\AmS{}math}

%
% D. AKTIONEN
% ---------------------------------------------------------------------------
%

%
%  1. Index erzeugen.
%
\makeindex

%
% E. SILBENTRENNUNG
% ---------------------------------------------------------------------------
%

\hyphenation{De-zi-mal-trenn-zeichen In-stal-la-ti-ons-as-sis-tent}

%
% G. BIBLIOGRAPHIE
% ---------------------------------------------------------------------------
%

\usepackage[backend = biber, style=numeric, sorting = none]{biblatex}
\addbibresource{bib/references.bib}


%
% F. BEFEHLE
% ---------------------------------------------------------------------------
%

\newcommand{\cob}{Care-O-bot\textsuperscript{\textregistered}}
\newcommand{\raw}{rob@work\textsuperscript{\textregistered}}


%
% F. SONSTIGE PACKAGES
% ---------------------------------------------------------------------------
%

\usepackage[printonlyused]{acronym}

% Todo Notes

\usepackage{todonotes}


% Grad Zeichen
\usepackage{textcomp}
\usepackage{units}
\DeclareUnicodeCharacter{B0}{\textdegree}




\usepackage{tikz}
\usetikzlibrary{shapes,arrows,positioning,fit,calc}

\usepackage{bm,times}

\usepackage{pstricks}
\usepackage{subfigure}
%
% ===========================================================================
% EOF
%

\usepackage{chngcntr}
\counterwithout{footnote}{chapter}
%\let\stdsection\section
%\renewcommand\section{\newpage\stdsection}
