\chapter{Aktueller Stand} \section{Das Robot Operating System}

\ac{ROS} ist ein Framework, das als Grundlage für Forschungs- und
Entwicklungsroboter dienen soll. \ac{ROS} wird als Open-Source Software von
Willow-Garage sowie einigen weiteren Einrichtungen entwickelt. Das Hauptziel bei
der Entwicklung von \ac{ROS} ist es, ein Framework zu schaffen, in dem
geschriebener Code nicht an Hardware gebunden ist, sondern wiederverwendet
werden kann. Dazu wird die Software für den Roboter in einzelne ``Nodes''
aufgeteilt, die jeweils eigene Aufgaben erfüllen. Nodes werden zu Packages
zusammengefasst, die einen größeren Funktionsblock einer Anwendung darstellen.
Beispielhaft hierfür ist das Package \texttt{cob\_calibration\_executive} mit
dem alle Bewegungen, die der Roboter während der Kalibrierung macht, gesteuert
werden. Alle für eine Anwendung notwendigen neuen Funktionsblöcke werden in
Stacks zusammengefasst. Die Änderungen und Erweiterungen die in dieser
Bachelorthesis gemacht wurden, betreffen hauptsächlich das Stack
\texttt{cob\_calibration} in dem alle Funktionen für die Kalibrierung abgelegt
sind. 

Die Kommunikation dieser Nodes kann durch drei verschiedene Systeme erfolgen.
Die drei Kommunikationswege zwischen den Nodes unterscheiden sich vorallem
durch die Funktion der beteiligten Kommunikationspartner. 

\begin{description}

  \item[\ac{ROS} Services] Durch die ``Services'' können zwei Nodes direkt
    miteinander kommunizieren. Hierbei sendet der Client-Node eine Anfrage an
    den Server-Node, in der bestimmte Parameter übergeben werden. Der
    Server-Node antwortet auf dem gleichen Weg.


  \item[\ac{ROS} Topics]Dem gegenüber gibt es die ``Topics'' in denen beliebig
    viele Nodes Nachrichten hinterlassen können. Die jeweils aktuellste
    Nachricht kann dann wiederum von beliebig vielen Nodes gelesen werden. Die
    Nachrichten enthalten keine Informationen von wem sie gesendet werden, also
    gibt es für den Empfänger keine Möglichkeit zu antworten oder zu quitieren.
    Dieses Verfahren wird zum Beispiel beim Verbreiten von Sensordaten
    eingesetzt.


  \item[\ac{ROS} Parameters]Als letztes wird der ``Parameter Server'' für die
    Speicherung von Daten verwendet. Hier können Daten hinterlegt werden, die
    immer den gleichen Wert haben und von mehreren Nodes benutzt werden. Ein
    Beispiel hierfür sind die Kameraparameter oder die Beschreibung des
    Roboters. 

\end{description}

Die Schnittstellen sind dabei so entwickelt, dass es problemlos möglich ist, die
Nodes auf unterschiedlichen vernetzten Computern auszuführen.  Um geschriebenen
Code auf anderen Robotern zu verwenden, muss also nur dafür gesorgt werden, dass
alle Kommunikationsschnittstellen vorhanden sind und richtig angewandt werden.
Durch die modulare Struktur von \ac{ROS} können viele vorhandene Lösungen für
die eigenen Aufgaben genutzt und angepasst werden.  Außerdem bietet \ac{ROS}
eine Simulationsumgebung und eine nahtlose Anbindung an andere Softwarepakete
wie OpenCV.

In \ref{fig:ros} ist der Ablauf der Datenaufnahme schematisch dargestellt. Der
\ac{ROS}-Node \allowbreak \texttt{collect\_data} schickt über \ac{ROS} eine Nachricht
an den Node \texttt{arm\_controller} der den am PC1 angeschlossenen 
Roboterarm steuert. Außerdem werden die Daten der am PC2 angeschlossenen Kamera
vom \texttt{camera\_driver} verarbeitet und über ein \ac{ROS}-Topic an den 
\texttt{collect\_data} Node weitergegeben.

\begin{figure}[Htbp]





% Define a few styles and constants
\tikzstyle{pc}=[draw, fill=blue!20, text width=5em, text centered, minimum height=2.5em, minimum width=20em]

\tikzstyle{node}=[pc, fill=black!20, text centered]

\tikzstyle{ros} = [pc, text width=10em, fill=red!20, 
    minimum height=3em, rounded corners]
    
\tikzstyle{hw} = [text centered]
\resizebox{\textwidth}{!}{
\begin{tikzpicture}[>=triangle 60]


\node [pc](pc1)  {PC1 (Ubuntu)};
\node [pc](pc2)[right=2of pc1] {PC2 (Ubuntu)};


\node [node](armc) [above=10em of pc1.west, anchor=west,  minimum width=9em]{arm\_controller};

\node [node](camd) [above=10em of pc2.west, anchor=west]{camera\_driver};


\node [node](coll) [above=10em of pc1.east, anchor=east,  minimum width=9em]{collect\_data};

\node [hw](cam) [below=4em of pc2.east, anchor=east]{Kamera};

\node [hw](arm) [below=4em of pc1.west, anchor=west]{Roboterarm};




\draw [<-](cam) -- (cam |- camd.south);
\draw [<-](arm) -- (arm |- armc.south);

\node (ros) [ros][above=5em of pc1, fit={(pc1) (pc2)}] {ROS Framework};

\draw (pc2) -- (pc2 |- ros.south);
\draw (pc1) -- (pc1 |- ros.south);
\draw (pc1) -- (pc2);
\let \p1 = (ros) 
\let \p2 = (pc1)
\draw [->]
 let 
 \p1 = (ros), 
 \p2 = (coll.south),
 \p3 = (armc.south)
 in
 (\x2-2em,\y2) -- (\x2-2em,\y1-1em) -- (\x3+2em,\y1-1em) -- (\x3+2em,\y3);
 
\draw [->]
 let 
 \p1 = (ros), 
 \p2 = (camd.south),
 \p3 = (coll.south)
 in
 (\x2-2em,\y2) -- (\x2-2em,\y1-1em) -- (\x3+2em,\y1-1em) -- (\x3+2em,\y3);



 
\end{tikzpicture}}



\label{fig:ros}
\caption{\ac{ROS} Architektur}


\end{figure}


Während der Bearbeitungszeit der Thesis wurde auf allen Computern die Version
``Electric Emys'' eingesetzt.

\section{Fähigkeiten und Funktionen des \protect\cob} % (fold)

\label{sec:Fähigkeiten und Funktionen des cob}

Zu Begin der Arbeit liegt bereits eine vollständige Beschreibung aller
relevanten Komponenten der zu kalibrierenden Roboter vor. Daraus lässt sich
sowohl die Vorwärtskinematik, also die Transformation eines Koordinatensystems
einer Komponente, beispielsweise der Befestigungspunkt der Roboterhand am
Roboterarm, zu einem anderen Koordinatensystem der selben Komponente zum
Beispiel der Befestigungspunkt des Arms an der Roboterplattform, anhand von
Gelenkwinkeln berechnen. Außerdem kann die inverse Kinematik, also die
benötigten Gelenkwinkel zum Erreichen eines bestimmten Punktes, für den Arm
berechnet werden.

% section Fähigkeiten und Funktionen des \cob (end)

\section{Kalibrierung}

Der Ausgangszustand der automatischen Kalibrierung lieferte für den \cob 3-3
gute und zuverlässige Ergebnisse. Zur Kalibrierung muss ein
Kalibrierungsmuster, das für den \cob bisher ein Schachbrettmuster mit neun mal
sechs inneren Ecken \cite{opencv_chessboard} ist, anstatt der \ac{SDH} an
den Arm montiert werden. Nach dem anschließenden Einstellen der Nullwerte für
die Aktoren beginn die erste Datenaufnahme für die Kalibrierung der
Stereokameras. Die Kameras Microsoft Kinect oder Asus XTion Pro müssen aufgrund der
guten Kalibrierung vom Hersteller nicht kalibriert werden. 

Mit den gewonnenen Daten kann dann der Kalibrieralgorithmus für Stereokameras,
der von OpenCV bereitgestellt wird sowohl die intrinsischen Kameraparameter als
auch den Versatz der Kameras zueinander, den sogenannten Baselineshift,
berechnen.

Mit den jetzt kalibrierten Kameras werden ein zweites mal Daten aufgenommen.
Hierzu werden der Roboterarm und der Torso in vorher festgelegte Positionen
gefahren. An jeder Position werden die aktuellen Gelenkwinkel zusammen mit der,
von den Kameras erkannten, Position des Kalibrierungsmusters aufgenommen.

Im letzten Schritt berechnet ein Optimierer anhand der aufgenommenen Daten die
gesuchten Transformationen \footnote{Siehe Kapitel \ref{sub:Kinematische Kalibrierung}}\cite{Haug2012}


%Die automatische Kalibrierung des \cob 3-6 konnte jedoch nicht erfolgreich
%ausgeführt werden. Grund dafür waren einige fest eingestellte Parameter und
%Konfigurationen, die für den \cob 3-6 angepasst werden mussten. Desweiteren
%sind die zu kalibrierenden Transformationen sowie Informationen zu den
%kinematischen Ketten in einer Konfigurationsdatei abgelegt. Diese
%Konfigurationsdatei musste manuell erstellt werden und mit Werten gefüllt
%werden, die aus der Beschreibung des Roboters berechnet wurden. Außerdem
%mussten für alle Aktoren und Manipulatoren die \ac{DH-Parameter} ermittelt und
%eingetragen werden.

Im folgenden werden die vorhanden Konfigurationsdateien mit den in ihnen
gespeicherten Parametern erklärt.


\begin{itemize}

  \item[\texttt{system.yaml} ] Die im Package \texttt{cob\_robot\_calibration}
    für den \cob 3-3 vorhandene \texttt{system.yaml} Datei enthält alle für die
    Berechnungen des Optimierers benötigten Daten. Dazu gehören:

    \begin{description}

      \item[Transformationen] Alle relevanten festen Transformationen zwischen
        dem \ac{baselink} und den Koordinatensystemen der Kameras und des
        Kalibrierungsmusters. Dabei werden sowohl bekannte als auch Näherungen
	für unbekannte Transformationen angegeben.

      \item[Beschreibungen der Aktoren] Die Berechnung der
        Vorwärtskinematik eines Aktors benötigt der Optimierer neben den
        \acl{DH-Parameter}n auch Informationen zum Übersetzungsverhältnis und
        zur Genauigkeit der einzelnen Gelenke. Das Übersetzungsverhältnis wird
        schon im Treiber des Aktors berücksichtigt und ist hier immer mit 1.0
        anzugeben.

      \item[Kameraparameter] Hier können Kameraparameter definiert werden. Da
        bei der vorliegenden Kalibrierung aber schon verarbeitete
        Bildinformationen eingesetzt werden sind hier nur Standardwerte
        eingetragen.

    \end{description}

  \item[\texttt{sensors.yaml}] In der Konfigurationsdatei werden die Ketten aus
    denen der Roboter zusammengesetzt werden kann definiert. Dazu wird in
    \texttt{camera\_chains} die die Ketten zu den Kameras angeben und 
    \texttt{chains}, die die kinematische Kette zum Kalibrierobjekt beschreiben
    unterschieden.
    Dazu wird jeder Kette eine eindeutige \texttt{sensor\_id} sowie eine
    Beschreibung der Transformation zum Endpunkt zugeordnet. Die Kette besteht
    aus einer festen Transformation vom Roboterursprung zum Ursprung des Aktors
    (\texttt{before\_chain}), aus der \texttt{chain\_id} zur Zuordnung einer in
    der \texttt{system.yaml} angegebenen Aktoren sowie deren Anzahl an Gelenken 
    und der \texttt{after\_chain} für die feste Transformation vom Ende des Aktors
    zum Ursprung des Sensors oder Kalibrierobjekts.

  \item[\texttt{free\_x.yaml}] Die drei \texttt{free\_x.yaml} Dateien haben den 
    gleichen Aufbau wie die \texttt{system.yaml}. Sie werden benötigt um dem 
    Optimierer mitzuteilen, wann welche Parameter variabel sind.
\end{itemize}


\section{Koordinatentransformationen}

Ein wichtiges Thema bei der Robotik ist die Transformation eines 
Koordinatensystems in ein anderes. In der Robotik bekommt jedes
Objekt, das maßgeblich an einer Operation beteiligt ist, ein eigenes
Koordinatensystem, um seine Lage, bestehend aus Position und Orientierung
im Raum zu beschreiben. Dadurch lässt sich berechnen, wie ein Objekt gegriffen
werden muss oder Kollisionen vermieden werden. Die Transformation von einem 
Koordinatensystem in ein anderes, wie sie zum Beispiel vorkommt, wenn die 
Kameras (Koordinatensystem 1) ein Objekt (Koordinatensystem 2) erkennen, kann
durch eine Translation und eine Rotation dargestellt werden. Wenn die Lage der
Kameras relativ zur Basis des Roboters (Koordinatensystem 0) bekannt ist, ist
auch die Lage des Objekts im Koordinatensystem 0 bekannt. Für den \cob\ ist 
jedes Koordinatensystem ein Kind eines sogenannten Parent-Frames, das wiederum 
ein Koordinatensystem ist. Ein Parent-Frame kann mehrere Kinder haben, aber ein
Kind kann nur einen Parent-Frame haben. Außerdem sind alle Koordinatensysteme
des \cob\ orthonormale Koordinatensysteme. 

\subsection{Translation}
\label{sub:Translation}


Die Translation beschreibt nur die Position eines Punktes im Raum. Für die 
Koordinatentransformation ist dieser Punkt der Ursprung des neuen Koordinatensystems.
Der Punkt P wird im n-Dimensionalen durch einen Koordinatenvektor mit n 
Elementen angegeben. Räumliche Transformationen beschränken sich auf drei Dimensionen
und damit den Punkt $P= [p_x\\p_y\\p_z]^T$. Jedes Element $p_n$ gibt die Entfernung
entlang der n-ten Achse des Parent-Frames an.

Da für jeden Punkt festgelegt sein muss, in welchem Koordinatensystem $A$ er angegeben
ist, wird das Bezugskoordinatensystem in der Form $^AP$ angegeben.

\subsection{Rotation}
\label{sub:Rotation}

Da in der Robotik die Position eines Objektes selten ausreichend ist, wird zusätzlich 
die Orientierung benötigt. Dafür bekommt das Objekt ein Koordinatensystem bestehend
aus dem Ursprung $P$ und drei Achsen. Die Ausrichtung der drei Achsen werden durch
drei Vektoren im Parent-Frame angegeben, die orthogonal sind und die Länge $1$
haben. Die drei Vektoren, die die Achsen des Koordinatensystems $B$ im Parent-Frame
$A$ angeben, werden eindeutig mit $^A\hat{X}_B$, $^A\hat{Y}_B$ und $^A\hat{Z}_B$
angegeben. Diese drei Vektoren werden zu der sogenannten Rotationsmatrix
\begin{equation}
  ^A_BR = [ ^A\hat{X}_B ^A\hat{Y}_B ^A\hat{Z}_B ] = \begin{pmatrix}
    r_{11}&r_{12}&r_{13}\\
    r_{21}&r_{22}&r_{23}\\
    r_{31}&r_{32}&r_{33}
  \end{pmatrix}
  \label{eq:rotationsmatrix}
\end{equation}
zusammengefasst.

Für die Rotation $^B_AR$ gilt wie in \cite{craig2005} angegeben:
\begin{equation}
  ^B_AR=^A_BR^T
\end{equation}


Zur vereinfachten Darstellung von Rotationen genügen drei Winkel, aus denen die
Rotationsmatrix berechnet werden kann. Am \cob\ wird hierfür die \begin{quote}Roll-Pitch-Yaw\end{quote}
Präsentation gewählt. Weitere verbreitete Methoden, sind wie in \cite{sciavicco2000modelling} 
Dargestellt die zwölf Versionen der Euler-Winkel.

\subsubsection{Roll-Pitch-Yaw}
\label{ssub:Roll-Pitch-Yaw}

Die Roll-Pitch-Yaw Beschreibung stammt aus der Luft- und Seefahrt und wird genutzt
um die Orientierung eines Objekts mit den Winkeln Rollwinkel, Gierwinkel und 
Nickwinkel anzugeben.
\begin{description}
  \item[$\phi$]Rollwinkel: Winkel, der eine Drehung um die X-Achse des Referenzkoordinatensystems
    angibt
  \item[$\psi$]Gierwinkel: Winkel, der die Drehung um die Z-Achse eines Referenzkoordinatensystems
    beschreibt
  \item[$\theta$]Nickwinkel: Winkel, der die Rotation um die feste Y-Achse beschreibt.

\end{description}


Die Rotationsmatrix berechnet sich daraus mit $c_x=cos(x)$ und $s_x=sin(x)$ zu:

\begin{equation}
  R_{RPY}=
  \begin{pmatrix}
    c_\phi*c_\theta     
    & c_\phi*s_\psi*s_\theta-s_\phi*c_\theta
    & c_\phi*s_\psi*c_\theta-s_\phi*s_\theta\\

    s_\phi*s_\psi          
    & s_\phi*s_\psi*s_\theta+c_\phi*c_\theta
    & s_\phi*s_\psi*c_\theta-c_\phi*s_\theta\\
  
    -s_\psi    
    & c_\psi*s_\theta   
    & c_\phi*c_\theta
  
  \end{pmatrix}
\end{equation}

Alle für den \cob\ angegebenen Orientierungen in der Roboterbeschreibung sind 
als Roll-Pitch-Yaw Winkel angegeben. Außerdem kann durch vorhandene Nodes 
die Orientierung von jedem Koordinatensystem des \cob\ relativ zu jedem 
anderen Koordinatensystem des \cob\ in einem Translationsvektor und den Roll-Pitch-Yaw
Winkeln berechnet werden.


\subsubsection{Denavit-Hartenberg Parameter}
\label{ssub:Denavit-Hartenberg Parameter}
\label{dh-p}

Die Denavit-Hartenberg Notation basiert auf einem von Denavit und Hartenberg 
entwickelten Verfahren zur Berechnung von Transformationen entlang einer 
kinematischen Kette. Dazu werden für jedes 
Gelenk vier Parameter benötigt. Die sogenannten \ac{DH-Parameter}. Aus diesen
Parametern kann dann für jedes Gelenk die Transformationsmatrix zum vorherigen 
Gelenk berechnet werden. Das Verfahren vereinfacht vor allem die Berechnung 
der Vorwärtskinematik, also die Berechnung von Positionen anhand der 
Gelenkzustände. 

Um eine Transformation mit vier statt den sonst üblichen 
sechs Parametern zu beschreiben, werden die möglichen Freiheitsgrade der Gelenke
limitiert. Um einen Aktor mit \ac{DH-Parameter}n zu beschreiben dürfen alle
Gelenke nur einen Freiheitsgrad haben. In der Anwendung führt dies aufgrund der gebräuchlichen Roboterbauformen
mit entweder Schubgelenken oder Drehgelenken aber selten zu Einschränkungen.

Zur Angabe einer Transformation werden die 
Parameter $[a, \alpha, d, \Theta]$ angegeben. Daraus kann die homogene 
Translationsmatrix berechnet werden.\cite{craig2005}



\section{Kalibrieralgorithmen} % (fold)

\label{sub:Kalibrieralgorithmen}


Zu
Beginn der Arbeit gab es bereits einige Algorithmen zur automatischen
Kalibrierung der am Roboter eingesetzten Hardwarekomponenten. Die wichtigsten
und bereits am \cob implementierten sind die Verfahren zur Kalibrierung von
Mono- und Stereokameras, die OpenCV \footnote{Open Computer Vision: Eine 
Open-Source Softwarebibliothek die Funktionen zur Bildverarbeitung bereitstellt}
zur Verfügung stellt und das von Vijay
Pradeep entwickelte Verfahren zur Kalibrierung von Robotern mit mehreren Armen
und mehreren Sensoren.\cite{Pradeep2010}


\subsection{Kamerakalibrierung} % (fold)

\label{sub:Kamerakalibrierung}


Um
Objekte, die von den Kameras erkannt werden, in der dreidimensionalen Welt
einzuordnen, muss ein möglichst exaktes mathematisches Modell entwickelt werden,
nach dem Objektpunkte auf die Sensorfläche projiziert werden. Außerdem müssen
die Parameter des Stereokamerasystems für die Bestimmung von Tiefendaten aus
zwei Einzelbildern bekannt sein. Da diese Parameter nicht am
Roboter oder der Kamera abgemessen werden können, wird auf Methoden der
Bildverarbeitung zurückgegriffen.

\subsubsection{Monokamerakalibrierung} % (fold)

\label{ssub:Monokamera}


Zur Berechnung der Projektion eines Punktes im Raum auf einen Film oder
Bildsensor muss ein Modell erzeugt werden, das die Abbildungseigenschaften
einer Kamera möglichst einfach, aber genau genug wiedergibt. Dieses Modell kann
durch eine Kombination des sogenannten Lochkameramodells und des modellierten
Einflusses von optischen Linsen beschrieben werden \cite{Bradski2008}.

Die Lochkamera besteht, wie in Abbildung \ref{fig:Lochkamera} aus einer 
geschlossenen Box, in der auf einer Seite ein
Sensor oder Film angebracht ist. Gegenüber vom Sensor befindet sich ein kleines
Loch, das optische Zentrum C, durch das idealerweise nur ein Lichtstrahl pro
Objektpunkt auf den Sensor
trifft. Gemäß den geometrischen Gesetzen ergibt sich ein auf dem Kopf stehendes
Abbild des Objektes auf der Bildebene oder dem Sensor. Zur Vereinfachung der
Rechnung kann eine virtuelle Bildebene vor der Kamera angenommen werden, die
durch eine Punktspiegelung der Bildebene am optischen Zentrum erzeugt werden
kann. Im Vergleich zur Berechnung für die Bildebene sind die Werte hierbei
invertiert\cite{forsyth2011}.


\begin{figure}[htpb]
  \centering
  \def\svgwidth{\textwidth}
  %LaTeX with PSTricks extensions
%%Creator: inkscape 0.48.3.1
%%Please note this file requires PSTricks extensions
\psset{xunit=.5pt,yunit=.5pt,runit=.5pt}
\begin{pspicture}(744.09448242,1052.36218262)
{
\newrgbcolor{curcolor}{0 0 0}
\pscustom[linewidth=1.0092119,linecolor=curcolor]
{
\newpath
\moveto(250.5,349.49075262)
\lineto(250.5,100.50925262)
}
}
{
\newrgbcolor{curcolor}{0 0 0}
\pscustom[linewidth=1.0092119,linecolor=curcolor]
{
\newpath
\moveto(499.49075,100.50000262)
\lineto(250.50925,100.50000262)
}
}
{
\newrgbcolor{curcolor}{0 0 0}
\pscustom[linewidth=1.0092119,linecolor=curcolor]
{
\newpath
\moveto(499.50461,349.49075262)
\lineto(499.50461,100.50925262)
}
}
{
\newrgbcolor{curcolor}{0 0 0}
\pscustom[linewidth=0.69921553,linecolor=curcolor]
{
\newpath
\moveto(499.86517,349.50460262)
\lineto(380.34961,349.50460262)
}
}
{
\newrgbcolor{curcolor}{0 0 0}
\pscustom[linewidth=0.69921553,linecolor=curcolor]
{
\newpath
\moveto(369.86517,349.50460262)
\lineto(250.34961,349.50460262)
}
}
{
\newrgbcolor{curcolor}{1 0 0}
\pscustom[linewidth=1.029109,linecolor=curcolor]
{
\newpath
\moveto(260.51455,105.51455262)
\lineto(489.45549,105.51455262)
}
}
{
\newrgbcolor{curcolor}{0 0 0}
\pscustom[linewidth=1.44091034,linecolor=curcolor,strokeopacity=0.48692814,linestyle=dashed,dash=1.44091018 2.88182037]
{
\newpath
\moveto(375.5,962.19305962)
\lineto(375.5,94.95904262)
}
}
{
\newrgbcolor{curcolor}{0 0 0}
\pscustom[linestyle=none,fillstyle=solid,fillcolor=curcolor]
{
\newpath
\moveto(335.55349731,87.25756836)
\curveto(334.69411845,87.25756041)(334.01052538,86.93724823)(333.50271606,86.29663086)
\curveto(332.99880764,85.65599952)(332.74685477,84.78295351)(332.74685669,83.67749023)
\curveto(332.74685477,82.57592447)(332.99880764,81.70483159)(333.50271606,81.06420898)
\curveto(334.01052538,80.42358287)(334.69411845,80.10327069)(335.55349731,80.10327148)
\curveto(336.41286673,80.10327069)(337.09255355,80.42358287)(337.59255981,81.06420898)
\curveto(338.09645879,81.70483159)(338.34841167,82.57592447)(338.34841919,83.67749023)
\curveto(338.34841167,84.78295351)(338.09645879,85.65599952)(337.59255981,86.29663086)
\curveto(337.09255355,86.93724823)(336.41286673,87.25756041)(335.55349731,87.25756836)
\moveto(335.55349731,88.21850586)
\curveto(336.78005386,88.21849695)(337.76052163,87.80638799)(338.49490356,86.98217773)
\curveto(339.22927016,86.16185838)(339.59645729,85.06029699)(339.59646606,83.67749023)
\curveto(339.59645729,82.298581)(339.22927016,81.1970196)(338.49490356,80.37280273)
\curveto(337.76052163,79.55248999)(336.78005386,79.14233415)(335.55349731,79.14233398)
\curveto(334.32302507,79.14233415)(333.33865105,79.55248999)(332.60037231,80.37280273)
\curveto(331.86599627,81.19311335)(331.49880914,82.29467475)(331.49880981,83.67749023)
\curveto(331.49880914,85.06029699)(331.86599627,86.16185838)(332.60037231,86.98217773)
\curveto(333.33865105,87.80638799)(334.32302507,88.21849695)(335.55349731,88.21850586)
}
}
{
\newrgbcolor{curcolor}{0 0 0}
\pscustom[linestyle=none,fillstyle=solid,fillcolor=curcolor]
{
\newpath
\moveto(342.44412231,80.29663086)
\lineto(342.44412231,76.81616211)
\lineto(341.36013794,76.81616211)
\lineto(341.36013794,85.87475586)
\lineto(342.44412231,85.87475586)
\lineto(342.44412231,84.87866211)
\curveto(342.67068241,85.26928115)(342.95583838,85.55834336)(343.29959106,85.74584961)
\curveto(343.64724394,85.93724923)(344.06130602,86.03295226)(344.54177856,86.03295898)
\curveto(345.3386485,86.03295226)(345.98513222,85.71654633)(346.48123169,85.08374023)
\curveto(346.98122498,84.4509226)(347.23122473,83.61889218)(347.23123169,82.58764648)
\curveto(347.23122473,81.55639424)(346.98122498,80.72436382)(346.48123169,80.09155273)
\curveto(345.98513222,79.45874009)(345.3386485,79.14233415)(344.54177856,79.14233398)
\curveto(344.06130602,79.14233415)(343.64724394,79.23608406)(343.29959106,79.42358398)
\curveto(342.95583838,79.61498993)(342.67068241,79.90600527)(342.44412231,80.29663086)
\moveto(346.11209106,82.58764648)
\curveto(346.11208522,83.38061117)(345.94802289,84.00170429)(345.61990356,84.45092773)
\curveto(345.29567979,84.90404714)(344.84841461,85.13060942)(344.27810669,85.13061523)
\curveto(343.70779075,85.13060942)(343.25857245,84.90404714)(342.93045044,84.45092773)
\curveto(342.60622935,84.00170429)(342.44412014,83.38061117)(342.44412231,82.58764648)
\curveto(342.44412014,81.79467525)(342.60622935,81.171629)(342.93045044,80.71850586)
\curveto(343.25857245,80.26928615)(343.70779075,80.044677)(344.27810669,80.04467773)
\curveto(344.84841461,80.044677)(345.29567979,80.26928615)(345.61990356,80.71850586)
\curveto(345.94802289,81.171629)(346.11208522,81.79467525)(346.11209106,82.58764648)
}
}
{
\newrgbcolor{curcolor}{0 0 0}
\pscustom[linestyle=none,fillstyle=solid,fillcolor=curcolor]
{
\newpath
\moveto(350.08474731,87.73803711)
\lineto(350.08474731,85.87475586)
\lineto(352.30545044,85.87475586)
\lineto(352.30545044,85.03686523)
\lineto(350.08474731,85.03686523)
\lineto(350.08474731,81.47436523)
\curveto(350.08474512,80.93920736)(350.15701067,80.5954577)(350.30154419,80.44311523)
\curveto(350.44997913,80.29077051)(350.74880695,80.21459871)(351.19802856,80.21459961)
\lineto(352.30545044,80.21459961)
\lineto(352.30545044,79.31225586)
\lineto(351.19802856,79.31225586)
\curveto(350.36599484,79.31225586)(349.79177666,79.46655258)(349.47537231,79.77514648)
\curveto(349.15896479,80.08764571)(349.00076183,80.65405139)(349.00076294,81.47436523)
\lineto(349.00076294,85.03686523)
\lineto(348.20974731,85.03686523)
\lineto(348.20974731,85.87475586)
\lineto(349.00076294,85.87475586)
\lineto(349.00076294,87.73803711)
\lineto(350.08474731,87.73803711)
}
}
{
\newrgbcolor{curcolor}{0 0 0}
\pscustom[linestyle=none,fillstyle=solid,fillcolor=curcolor]
{
\newpath
\moveto(353.72927856,85.87475586)
\lineto(354.80740356,85.87475586)
\lineto(354.80740356,79.31225586)
\lineto(353.72927856,79.31225586)
\lineto(353.72927856,85.87475586)
\moveto(353.72927856,88.42944336)
\lineto(354.80740356,88.42944336)
\lineto(354.80740356,87.06420898)
\lineto(353.72927856,87.06420898)
\lineto(353.72927856,88.42944336)
}
}
{
\newrgbcolor{curcolor}{0 0 0}
\pscustom[linestyle=none,fillstyle=solid,fillcolor=curcolor]
{
\newpath
\moveto(361.24099731,85.68139648)
\lineto(361.24099731,84.66186523)
\curveto(360.9363048,84.81810973)(360.61989887,84.93529711)(360.29177856,85.01342773)
\curveto(359.96364953,85.09154696)(359.62380612,85.13060942)(359.27224731,85.13061523)
\curveto(358.73708825,85.13060942)(358.33474491,85.04857825)(358.06521606,84.88452148)
\curveto(357.79958919,84.72045358)(357.66677682,84.47436007)(357.66677856,84.14624023)
\curveto(357.66677682,83.89623565)(357.76247985,83.69897022)(357.95388794,83.55444336)
\curveto(358.14529197,83.41381426)(358.53005721,83.27904877)(359.10818481,83.15014648)
\lineto(359.47732544,83.06811523)
\curveto(360.24294612,82.90404914)(360.78591433,82.6716275)(361.10623169,82.37084961)
\curveto(361.43044494,82.07397185)(361.59255415,81.65795664)(361.59255981,81.12280273)
\curveto(361.59255415,80.51342653)(361.35036689,80.03100514)(360.86599731,79.67553711)
\curveto(360.38552411,79.32006835)(359.72341539,79.14233415)(358.87966919,79.14233398)
\curveto(358.52810409,79.14233415)(358.16091696,79.17749037)(357.77810669,79.24780273)
\curveto(357.39919897,79.31420898)(356.99880874,79.41577138)(356.57693481,79.55249023)
\lineto(356.57693481,80.66577148)
\curveto(356.97537127,80.45873909)(357.367949,80.30248924)(357.75466919,80.19702148)
\curveto(358.14138572,80.0954582)(358.52419784,80.044677)(358.90310669,80.04467773)
\curveto(359.41091571,80.044677)(359.80154031,80.13061442)(360.07498169,80.30249023)
\curveto(360.34841477,80.47827032)(360.48513338,80.72436382)(360.48513794,81.04077148)
\curveto(360.48513338,81.33373821)(360.38552411,81.55834736)(360.18630981,81.71459961)
\curveto(359.99099325,81.87084705)(359.55935306,82.02123753)(358.89138794,82.16577148)
\lineto(358.51638794,82.25366211)
\curveto(357.84841727,82.39428403)(357.36599588,82.60912756)(357.06912231,82.89819336)
\curveto(356.77224647,83.19115823)(356.62380912,83.59154846)(356.62380981,84.09936523)
\curveto(356.62380912,84.71654733)(356.8425589,85.19310935)(357.28005981,85.52905273)
\curveto(357.71755802,85.86498368)(358.33865115,86.03295226)(359.14334106,86.03295898)
\curveto(359.54177495,86.03295226)(359.91677457,86.00365542)(360.26834106,85.94506836)
\curveto(360.61989887,85.88646804)(360.9441173,85.7985775)(361.24099731,85.68139648)
}
}
{
\newrgbcolor{curcolor}{0 0 0}
\pscustom[linestyle=none,fillstyle=solid,fillcolor=curcolor]
{
\newpath
\moveto(368.03787231,85.62280273)
\lineto(368.03787231,84.61499023)
\curveto(367.73317927,84.78295351)(367.42653895,84.90795339)(367.11795044,84.98999023)
\curveto(366.81325831,85.07592197)(366.50466487,85.11889068)(366.19216919,85.11889648)
\curveto(365.49294713,85.11889068)(364.94997892,84.89623465)(364.56326294,84.45092773)
\curveto(364.1765422,84.00951679)(363.98318302,83.38842366)(363.98318481,82.58764648)
\curveto(363.98318302,81.78686276)(364.1765422,81.16381651)(364.56326294,80.71850586)
\curveto(364.94997892,80.27709864)(365.49294713,80.05639574)(366.19216919,80.05639648)
\curveto(366.50466487,80.05639574)(366.81325831,80.09741132)(367.11795044,80.17944336)
\curveto(367.42653895,80.26537991)(367.73317927,80.3923329)(368.03787231,80.56030273)
\lineto(368.03787231,79.56420898)
\curveto(367.73708551,79.42358387)(367.42458582,79.31811523)(367.10037231,79.24780273)
\curveto(366.78005522,79.17749037)(366.43825869,79.14233415)(366.07498169,79.14233398)
\curveto(365.08669754,79.14233415)(364.30154207,79.45288072)(363.71951294,80.07397461)
\curveto(363.13748074,80.69506698)(362.8464654,81.53295676)(362.84646606,82.58764648)
\curveto(362.8464654,83.65795464)(363.13943386,84.49975067)(363.72537231,85.11303711)
\curveto(364.31521393,85.72631195)(365.12185375,86.03295226)(366.14529419,86.03295898)
\curveto(366.47732115,86.03295226)(366.80153957,85.99779605)(367.11795044,85.92749023)
\curveto(367.43435144,85.86107744)(367.74099176,85.75951504)(368.03787231,85.62280273)
}
}
{
\newrgbcolor{curcolor}{0 0 0}
\pscustom[linestyle=none,fillstyle=solid,fillcolor=curcolor]
{
\newpath
\moveto(375.37966919,83.27319336)
\lineto(375.37966919,79.31225586)
\lineto(374.30154419,79.31225586)
\lineto(374.30154419,83.23803711)
\curveto(374.30153868,83.85912631)(374.18044505,84.3239696)(373.93826294,84.63256836)
\curveto(373.69607054,84.94115648)(373.33278965,85.0954532)(372.84841919,85.09545898)
\curveto(372.26638447,85.0954532)(371.80740055,84.90990651)(371.47146606,84.53881836)
\curveto(371.13552622,84.16771975)(370.96755764,83.66186088)(370.96755981,83.02124023)
\lineto(370.96755981,79.31225586)
\lineto(369.88357544,79.31225586)
\lineto(369.88357544,88.42944336)
\lineto(370.96755981,88.42944336)
\lineto(370.96755981,84.85522461)
\curveto(371.22536988,85.24974992)(371.52810396,85.5446715)(371.87576294,85.73999023)
\curveto(372.22732201,85.93529611)(372.63161848,86.03295226)(373.08865356,86.03295898)
\curveto(373.84255477,86.03295226)(374.4128667,85.7985775)(374.79959106,85.32983398)
\curveto(375.18630342,84.86498468)(375.3796626,84.17943849)(375.37966919,83.27319336)
}
}
{
\newrgbcolor{curcolor}{0 0 0}
\pscustom[linestyle=none,fillstyle=solid,fillcolor=curcolor]
{
\newpath
\moveto(383.15505981,82.86303711)
\lineto(383.15505981,82.33569336)
\lineto(378.19802856,82.33569336)
\curveto(378.24490173,81.59350358)(378.46755776,81.02709789)(378.86599731,80.63647461)
\curveto(379.26833821,80.24975492)(379.8269314,80.05639574)(380.54177856,80.05639648)
\curveto(380.95583652,80.05639574)(381.35622674,80.10717694)(381.74295044,80.20874023)
\curveto(382.13356972,80.31030174)(382.52028808,80.46264533)(382.90310669,80.66577148)
\lineto(382.90310669,79.64624023)
\curveto(382.51638183,79.48217756)(382.11989786,79.35717769)(381.71365356,79.27124023)
\curveto(381.30739867,79.18530286)(380.89528971,79.14233415)(380.47732544,79.14233398)
\curveto(379.43044742,79.14233415)(378.60037012,79.44702135)(377.98709106,80.05639648)
\curveto(377.3777151,80.66577013)(377.0730279,81.48998806)(377.07302856,82.52905273)
\curveto(377.0730279,83.60326719)(377.36209011,84.45482884)(377.94021606,85.08374023)
\curveto(378.5222452,85.71654633)(379.30544754,86.03295226)(380.28982544,86.03295898)
\curveto(381.17263318,86.03295226)(381.86989811,85.7477963)(382.38162231,85.17749023)
\curveto(382.89724083,84.61107869)(383.15505307,83.83959508)(383.15505981,82.86303711)
\moveto(382.07693481,83.17944336)
\curveto(382.06911666,83.76928265)(381.9031012,84.23998531)(381.57888794,84.59155273)
\curveto(381.25857059,84.9431096)(380.83278977,85.11889068)(380.30154419,85.11889648)
\curveto(379.6999784,85.11889068)(379.21755701,84.94896897)(378.85427856,84.60913086)
\curveto(378.49490148,84.26928215)(378.28787044,83.79076701)(378.23318481,83.17358398)
\lineto(382.07693481,83.17944336)
}
}
{
\newrgbcolor{curcolor}{0 0 0}
\pscustom[linestyle=none,fillstyle=solid,fillcolor=curcolor]
{
}
}
{
\newrgbcolor{curcolor}{0 0 0}
\pscustom[linestyle=none,fillstyle=solid,fillcolor=curcolor]
{
\newpath
\moveto(391.71560669,86.89428711)
\lineto(390.11013794,82.54077148)
\lineto(393.32693481,82.54077148)
\lineto(391.71560669,86.89428711)
\moveto(391.04763794,88.06030273)
\lineto(392.38943481,88.06030273)
\lineto(395.72341919,79.31225586)
\lineto(394.49295044,79.31225586)
\lineto(393.69607544,81.55639648)
\lineto(389.75271606,81.55639648)
\lineto(388.95584106,79.31225586)
\lineto(387.70779419,79.31225586)
\lineto(391.04763794,88.06030273)
}
}
{
\newrgbcolor{curcolor}{0 0 0}
\pscustom[linestyle=none,fillstyle=solid,fillcolor=curcolor]
{
\newpath
\moveto(401.45974731,85.62280273)
\lineto(401.45974731,84.61499023)
\curveto(401.15505427,84.78295351)(400.84841395,84.90795339)(400.53982544,84.98999023)
\curveto(400.23513331,85.07592197)(399.92653987,85.11889068)(399.61404419,85.11889648)
\curveto(398.91482213,85.11889068)(398.37185392,84.89623465)(397.98513794,84.45092773)
\curveto(397.5984172,84.00951679)(397.40505802,83.38842366)(397.40505981,82.58764648)
\curveto(397.40505802,81.78686276)(397.5984172,81.16381651)(397.98513794,80.71850586)
\curveto(398.37185392,80.27709864)(398.91482213,80.05639574)(399.61404419,80.05639648)
\curveto(399.92653987,80.05639574)(400.23513331,80.09741132)(400.53982544,80.17944336)
\curveto(400.84841395,80.26537991)(401.15505427,80.3923329)(401.45974731,80.56030273)
\lineto(401.45974731,79.56420898)
\curveto(401.15896051,79.42358387)(400.84646082,79.31811523)(400.52224731,79.24780273)
\curveto(400.20193022,79.17749037)(399.86013369,79.14233415)(399.49685669,79.14233398)
\curveto(398.50857254,79.14233415)(397.72341707,79.45288072)(397.14138794,80.07397461)
\curveto(396.55935574,80.69506698)(396.2683404,81.53295676)(396.26834106,82.58764648)
\curveto(396.2683404,83.65795464)(396.56130886,84.49975067)(397.14724731,85.11303711)
\curveto(397.73708893,85.72631195)(398.54372875,86.03295226)(399.56716919,86.03295898)
\curveto(399.89919615,86.03295226)(400.22341457,85.99779605)(400.53982544,85.92749023)
\curveto(400.85622644,85.86107744)(401.16286676,85.75951504)(401.45974731,85.62280273)
}
}
{
\newrgbcolor{curcolor}{0 0 0}
\pscustom[linestyle=none,fillstyle=solid,fillcolor=curcolor]
{
\newpath
\moveto(408.80154419,83.27319336)
\lineto(408.80154419,79.31225586)
\lineto(407.72341919,79.31225586)
\lineto(407.72341919,83.23803711)
\curveto(407.72341368,83.85912631)(407.60232005,84.3239696)(407.36013794,84.63256836)
\curveto(407.11794554,84.94115648)(406.75466465,85.0954532)(406.27029419,85.09545898)
\curveto(405.68825947,85.0954532)(405.22927555,84.90990651)(404.89334106,84.53881836)
\curveto(404.55740122,84.16771975)(404.38943264,83.66186088)(404.38943481,83.02124023)
\lineto(404.38943481,79.31225586)
\lineto(403.30545044,79.31225586)
\lineto(403.30545044,88.42944336)
\lineto(404.38943481,88.42944336)
\lineto(404.38943481,84.85522461)
\curveto(404.64724488,85.24974992)(404.94997896,85.5446715)(405.29763794,85.73999023)
\curveto(405.64919701,85.93529611)(406.05349348,86.03295226)(406.51052856,86.03295898)
\curveto(407.26442977,86.03295226)(407.8347417,85.7985775)(408.22146606,85.32983398)
\curveto(408.60817842,84.86498468)(408.8015376,84.17943849)(408.80154419,83.27319336)
}
}
{
\newrgbcolor{curcolor}{0 0 0}
\pscustom[linestyle=none,fillstyle=solid,fillcolor=curcolor]
{
\newpath
\moveto(415.14724731,85.68139648)
\lineto(415.14724731,84.66186523)
\curveto(414.8425548,84.81810973)(414.52614887,84.93529711)(414.19802856,85.01342773)
\curveto(413.86989953,85.09154696)(413.53005612,85.13060942)(413.17849731,85.13061523)
\curveto(412.64333825,85.13060942)(412.24099491,85.04857825)(411.97146606,84.88452148)
\curveto(411.70583919,84.72045358)(411.57302682,84.47436007)(411.57302856,84.14624023)
\curveto(411.57302682,83.89623565)(411.66872985,83.69897022)(411.86013794,83.55444336)
\curveto(412.05154197,83.41381426)(412.43630721,83.27904877)(413.01443481,83.15014648)
\lineto(413.38357544,83.06811523)
\curveto(414.14919612,82.90404914)(414.69216433,82.6716275)(415.01248169,82.37084961)
\curveto(415.33669494,82.07397185)(415.49880415,81.65795664)(415.49880981,81.12280273)
\curveto(415.49880415,80.51342653)(415.25661689,80.03100514)(414.77224731,79.67553711)
\curveto(414.29177411,79.32006835)(413.62966539,79.14233415)(412.78591919,79.14233398)
\curveto(412.43435409,79.14233415)(412.06716696,79.17749037)(411.68435669,79.24780273)
\curveto(411.30544897,79.31420898)(410.90505874,79.41577138)(410.48318481,79.55249023)
\lineto(410.48318481,80.66577148)
\curveto(410.88162127,80.45873909)(411.274199,80.30248924)(411.66091919,80.19702148)
\curveto(412.04763572,80.0954582)(412.43044784,80.044677)(412.80935669,80.04467773)
\curveto(413.31716571,80.044677)(413.70779031,80.13061442)(413.98123169,80.30249023)
\curveto(414.25466477,80.47827032)(414.39138338,80.72436382)(414.39138794,81.04077148)
\curveto(414.39138338,81.33373821)(414.29177411,81.55834736)(414.09255981,81.71459961)
\curveto(413.89724325,81.87084705)(413.46560306,82.02123753)(412.79763794,82.16577148)
\lineto(412.42263794,82.25366211)
\curveto(411.75466727,82.39428403)(411.27224588,82.60912756)(410.97537231,82.89819336)
\curveto(410.67849647,83.19115823)(410.53005912,83.59154846)(410.53005981,84.09936523)
\curveto(410.53005912,84.71654733)(410.7488089,85.19310935)(411.18630981,85.52905273)
\curveto(411.62380802,85.86498368)(412.24490115,86.03295226)(413.04959106,86.03295898)
\curveto(413.44802495,86.03295226)(413.82302457,86.00365542)(414.17459106,85.94506836)
\curveto(414.52614887,85.88646804)(414.8503673,85.7985775)(415.14724731,85.68139648)
}
}
{
\newrgbcolor{curcolor}{0 0 0}
\pscustom[linestyle=none,fillstyle=solid,fillcolor=curcolor]
{
\newpath
\moveto(422.83474731,82.86303711)
\lineto(422.83474731,82.33569336)
\lineto(417.87771606,82.33569336)
\curveto(417.92458923,81.59350358)(418.14724526,81.02709789)(418.54568481,80.63647461)
\curveto(418.94802571,80.24975492)(419.5066189,80.05639574)(420.22146606,80.05639648)
\curveto(420.63552402,80.05639574)(421.03591424,80.10717694)(421.42263794,80.20874023)
\curveto(421.81325722,80.31030174)(422.19997558,80.46264533)(422.58279419,80.66577148)
\lineto(422.58279419,79.64624023)
\curveto(422.19606933,79.48217756)(421.79958536,79.35717769)(421.39334106,79.27124023)
\curveto(420.98708617,79.18530286)(420.57497721,79.14233415)(420.15701294,79.14233398)
\curveto(419.11013492,79.14233415)(418.28005762,79.44702135)(417.66677856,80.05639648)
\curveto(417.0574026,80.66577013)(416.7527154,81.48998806)(416.75271606,82.52905273)
\curveto(416.7527154,83.60326719)(417.04177761,84.45482884)(417.61990356,85.08374023)
\curveto(418.2019327,85.71654633)(418.98513504,86.03295226)(419.96951294,86.03295898)
\curveto(420.85232068,86.03295226)(421.54958561,85.7477963)(422.06130981,85.17749023)
\curveto(422.57692833,84.61107869)(422.83474057,83.83959508)(422.83474731,82.86303711)
\moveto(421.75662231,83.17944336)
\curveto(421.74880416,83.76928265)(421.5827887,84.23998531)(421.25857544,84.59155273)
\curveto(420.93825809,84.9431096)(420.51247727,85.11889068)(419.98123169,85.11889648)
\curveto(419.3796659,85.11889068)(418.89724451,84.94896897)(418.53396606,84.60913086)
\curveto(418.17458898,84.26928215)(417.96755794,83.79076701)(417.91287231,83.17358398)
\lineto(421.75662231,83.17944336)
}
}
{
\newrgbcolor{curcolor}{0 0 0}
\pscustom[linestyle=none,fillstyle=solid,fillcolor=curcolor]
{
\newpath
\moveto(508.95278931,106.21844482)
\lineto(508.95278931,103.0133667)
\lineto(510.85122681,103.0133667)
\curveto(511.48794066,103.01336573)(511.95864331,103.14422497)(512.26333618,103.40594482)
\curveto(512.57192395,103.67156819)(512.72622067,104.07586466)(512.72622681,104.61883545)
\curveto(512.72622067,105.16570732)(512.57192395,105.56805067)(512.26333618,105.8258667)
\curveto(511.95864331,106.0875814)(511.48794066,106.21844065)(510.85122681,106.21844482)
\lineto(508.95278931,106.21844482)
\moveto(508.95278931,109.81610107)
\lineto(508.95278931,107.17938232)
\lineto(510.70474243,107.17938232)
\curveto(511.28286274,107.17937719)(511.71254981,107.28679895)(511.99380493,107.50164795)
\curveto(512.27895549,107.72039227)(512.42153348,108.05242319)(512.42153931,108.4977417)
\curveto(512.42153348,108.93914105)(512.27895549,109.26921885)(511.99380493,109.48797607)
\curveto(511.71254981,109.70671841)(511.28286274,109.8160933)(510.70474243,109.81610107)
\lineto(508.95278931,109.81610107)
\moveto(507.76919556,110.78875732)
\lineto(510.79263306,110.78875732)
\curveto(511.6949717,110.78874858)(512.39028351,110.60124876)(512.87857056,110.22625732)
\curveto(513.36684503,109.85124951)(513.61098541,109.31804692)(513.61099243,108.62664795)
\curveto(513.61098541,108.09148565)(513.48598554,107.66570482)(513.23599243,107.3493042)
\curveto(512.98598604,107.03289296)(512.6187989,106.83562753)(512.13442993,106.75750732)
\curveto(512.71645506,106.63250273)(513.16762648,106.37078424)(513.48794556,105.97235107)
\curveto(513.81215709,105.57781629)(513.9742663,105.08367616)(513.97427368,104.4899292)
\curveto(513.9742663,103.70867753)(513.70864156,103.10516251)(513.17739868,102.67938232)
\curveto(512.64614263,102.25360086)(511.89028401,102.04071045)(510.90982056,102.04071045)
\lineto(507.76919556,102.04071045)
\lineto(507.76919556,110.78875732)
}
}
{
\newrgbcolor{curcolor}{0 0 0}
\pscustom[linestyle=none,fillstyle=solid,fillcolor=curcolor]
{
\newpath
\moveto(515.94888306,108.60321045)
\lineto(517.02700806,108.60321045)
\lineto(517.02700806,102.04071045)
\lineto(515.94888306,102.04071045)
\lineto(515.94888306,108.60321045)
\moveto(515.94888306,111.15789795)
\lineto(517.02700806,111.15789795)
\lineto(517.02700806,109.79266357)
\lineto(515.94888306,109.79266357)
\lineto(515.94888306,111.15789795)
}
}
{
\newrgbcolor{curcolor}{0 0 0}
\pscustom[linestyle=none,fillstyle=solid,fillcolor=curcolor]
{
\newpath
\moveto(519.27700806,111.15789795)
\lineto(520.35513306,111.15789795)
\lineto(520.35513306,102.04071045)
\lineto(519.27700806,102.04071045)
\lineto(519.27700806,111.15789795)
}
}
{
\newrgbcolor{curcolor}{0 0 0}
\pscustom[linestyle=none,fillstyle=solid,fillcolor=curcolor]
{
\newpath
\moveto(526.92349243,107.6071167)
\lineto(526.92349243,111.15789795)
\lineto(528.00161743,111.15789795)
\lineto(528.00161743,102.04071045)
\lineto(526.92349243,102.04071045)
\lineto(526.92349243,103.02508545)
\curveto(526.69692471,102.63445986)(526.40981562,102.34344452)(526.06216431,102.15203857)
\curveto(525.71841006,101.96453865)(525.30434798,101.87078874)(524.81997681,101.87078857)
\curveto(524.0270055,101.87078874)(523.38052178,102.18719468)(522.88052368,102.82000732)
\curveto(522.38442902,103.45281841)(522.13638239,104.28484883)(522.13638306,105.31610107)
\curveto(522.13638239,106.34734677)(522.38442902,107.17937719)(522.88052368,107.81219482)
\curveto(523.38052178,108.44500092)(524.0270055,108.76140685)(524.81997681,108.76141357)
\curveto(525.30434798,108.76140685)(525.71841006,108.66570382)(526.06216431,108.4743042)
\curveto(526.40981562,108.28679795)(526.69692471,107.99773574)(526.92349243,107.6071167)
\moveto(523.24966431,105.31610107)
\curveto(523.24966253,104.52312984)(523.41177174,103.90008359)(523.73599243,103.44696045)
\curveto(524.06411484,102.99774074)(524.51333314,102.77313159)(525.08364868,102.77313232)
\curveto(525.653957,102.77313159)(526.1031753,102.99774074)(526.43130493,103.44696045)
\curveto(526.75942465,103.90008359)(526.92348698,104.52312984)(526.92349243,105.31610107)
\curveto(526.92348698,106.10906576)(526.75942465,106.73015888)(526.43130493,107.17938232)
\curveto(526.1031753,107.63250173)(525.653957,107.85906401)(525.08364868,107.85906982)
\curveto(524.51333314,107.85906401)(524.06411484,107.63250173)(523.73599243,107.17938232)
\curveto(523.41177174,106.73015888)(523.24966253,106.10906576)(523.24966431,105.31610107)
}
}
{
\newrgbcolor{curcolor}{0 0 0}
\pscustom[linestyle=none,fillstyle=solid,fillcolor=curcolor]
{
\newpath
\moveto(535.83560181,105.5914917)
\lineto(535.83560181,105.06414795)
\lineto(530.87857056,105.06414795)
\curveto(530.92544372,104.32195817)(531.14809975,103.75555248)(531.54653931,103.3649292)
\curveto(531.9488802,102.97820951)(532.50747339,102.78485033)(533.22232056,102.78485107)
\curveto(533.63637851,102.78485033)(534.03676874,102.83563153)(534.42349243,102.93719482)
\curveto(534.81411171,103.03875633)(535.20083007,103.19109992)(535.58364868,103.39422607)
\lineto(535.58364868,102.37469482)
\curveto(535.19692383,102.21063215)(534.80043985,102.08563228)(534.39419556,101.99969482)
\curveto(533.98794066,101.91375745)(533.5758317,101.87078874)(533.15786743,101.87078857)
\curveto(532.11098941,101.87078874)(531.28091212,102.17547594)(530.66763306,102.78485107)
\curveto(530.05825709,103.39422472)(529.75356989,104.21844265)(529.75357056,105.25750732)
\curveto(529.75356989,106.33172178)(530.04263211,107.18328343)(530.62075806,107.81219482)
\curveto(531.2027872,108.44500092)(531.98598954,108.76140685)(532.97036743,108.76141357)
\curveto(533.85317517,108.76140685)(534.5504401,108.47625089)(535.06216431,107.90594482)
\curveto(535.57778282,107.33953328)(535.83559506,106.56804967)(535.83560181,105.5914917)
\moveto(534.75747681,105.90789795)
\curveto(534.74965865,106.49773724)(534.58364319,106.9684399)(534.25942993,107.32000732)
\curveto(533.93911258,107.67156419)(533.51333176,107.84734527)(532.98208618,107.84735107)
\curveto(532.38052039,107.84734527)(531.898099,107.67742356)(531.53482056,107.33758545)
\curveto(531.17544347,106.99773674)(530.96841243,106.5192216)(530.91372681,105.90203857)
\lineto(534.75747681,105.90789795)
}
}
{
\newrgbcolor{curcolor}{0 0 0}
\pscustom[linestyle=none,fillstyle=solid,fillcolor=curcolor]
{
\newpath
\moveto(542.31607056,105.31610107)
\curveto(542.31606471,106.10906576)(542.15200238,106.73015888)(541.82388306,107.17938232)
\curveto(541.49965928,107.63250173)(541.0523941,107.85906401)(540.48208618,107.85906982)
\curveto(539.91177024,107.85906401)(539.46255194,107.63250173)(539.13442993,107.17938232)
\curveto(538.81020885,106.73015888)(538.64809963,106.10906576)(538.64810181,105.31610107)
\curveto(538.64809963,104.52312984)(538.81020885,103.90008359)(539.13442993,103.44696045)
\curveto(539.46255194,102.99774074)(539.91177024,102.77313159)(540.48208618,102.77313232)
\curveto(541.0523941,102.77313159)(541.49965928,102.99774074)(541.82388306,103.44696045)
\curveto(542.15200238,103.90008359)(542.31606471,104.52312984)(542.31607056,105.31610107)
\moveto(538.64810181,107.6071167)
\curveto(538.87466191,107.99773574)(539.15981787,108.28679795)(539.50357056,108.4743042)
\curveto(539.85122343,108.66570382)(540.26528552,108.76140685)(540.74575806,108.76141357)
\curveto(541.54262799,108.76140685)(542.18911172,108.44500092)(542.68521118,107.81219482)
\curveto(543.18520447,107.17937719)(543.43520422,106.34734677)(543.43521118,105.31610107)
\curveto(543.43520422,104.28484883)(543.18520447,103.45281841)(542.68521118,102.82000732)
\curveto(542.18911172,102.18719468)(541.54262799,101.87078874)(540.74575806,101.87078857)
\curveto(540.26528552,101.87078874)(539.85122343,101.96453865)(539.50357056,102.15203857)
\curveto(539.15981787,102.34344452)(538.87466191,102.63445986)(538.64810181,103.02508545)
\lineto(538.64810181,102.04071045)
\lineto(537.56411743,102.04071045)
\lineto(537.56411743,111.15789795)
\lineto(538.64810181,111.15789795)
\lineto(538.64810181,107.6071167)
}
}
{
\newrgbcolor{curcolor}{0 0 0}
\pscustom[linestyle=none,fillstyle=solid,fillcolor=curcolor]
{
\newpath
\moveto(550.83560181,105.5914917)
\lineto(550.83560181,105.06414795)
\lineto(545.87857056,105.06414795)
\curveto(545.92544372,104.32195817)(546.14809975,103.75555248)(546.54653931,103.3649292)
\curveto(546.9488802,102.97820951)(547.50747339,102.78485033)(548.22232056,102.78485107)
\curveto(548.63637851,102.78485033)(549.03676874,102.83563153)(549.42349243,102.93719482)
\curveto(549.81411171,103.03875633)(550.20083007,103.19109992)(550.58364868,103.39422607)
\lineto(550.58364868,102.37469482)
\curveto(550.19692383,102.21063215)(549.80043985,102.08563228)(549.39419556,101.99969482)
\curveto(548.98794066,101.91375745)(548.5758317,101.87078874)(548.15786743,101.87078857)
\curveto(547.11098941,101.87078874)(546.28091212,102.17547594)(545.66763306,102.78485107)
\curveto(545.05825709,103.39422472)(544.75356989,104.21844265)(544.75357056,105.25750732)
\curveto(544.75356989,106.33172178)(545.04263211,107.18328343)(545.62075806,107.81219482)
\curveto(546.2027872,108.44500092)(546.98598954,108.76140685)(547.97036743,108.76141357)
\curveto(548.85317517,108.76140685)(549.5504401,108.47625089)(550.06216431,107.90594482)
\curveto(550.57778282,107.33953328)(550.83559506,106.56804967)(550.83560181,105.5914917)
\moveto(549.75747681,105.90789795)
\curveto(549.74965865,106.49773724)(549.58364319,106.9684399)(549.25942993,107.32000732)
\curveto(548.93911258,107.67156419)(548.51333176,107.84734527)(547.98208618,107.84735107)
\curveto(547.38052039,107.84734527)(546.898099,107.67742356)(546.53482056,107.33758545)
\curveto(546.17544347,106.99773674)(545.96841243,106.5192216)(545.91372681,105.90203857)
\lineto(549.75747681,105.90789795)
}
}
{
\newrgbcolor{curcolor}{0 0 0}
\pscustom[linestyle=none,fillstyle=solid,fillcolor=curcolor]
{
\newpath
\moveto(558.06021118,106.00164795)
\lineto(558.06021118,102.04071045)
\lineto(556.98208618,102.04071045)
\lineto(556.98208618,105.9664917)
\curveto(556.98208067,106.5875809)(556.86098704,107.05242419)(556.61880493,107.36102295)
\curveto(556.37661253,107.66961107)(556.01333164,107.82390779)(555.52896118,107.82391357)
\curveto(554.94692646,107.82390779)(554.48794254,107.6383611)(554.15200806,107.26727295)
\curveto(553.81606821,106.89617434)(553.64809963,106.39031547)(553.64810181,105.74969482)
\lineto(553.64810181,102.04071045)
\lineto(552.56411743,102.04071045)
\lineto(552.56411743,108.60321045)
\lineto(553.64810181,108.60321045)
\lineto(553.64810181,107.5836792)
\curveto(553.90591188,107.97820451)(554.20864595,108.27312609)(554.55630493,108.46844482)
\curveto(554.907864,108.6637507)(555.31216047,108.76140685)(555.76919556,108.76141357)
\curveto(556.52309676,108.76140685)(557.09340869,108.52703209)(557.48013306,108.05828857)
\curveto(557.86684541,107.59343927)(558.0602046,106.90789308)(558.06021118,106.00164795)
}
}
{
\newrgbcolor{curcolor}{0 0 0}
\pscustom[linestyle=none,fillstyle=solid,fillcolor=curcolor]
{
\newpath
\moveto(565.83560181,105.5914917)
\lineto(565.83560181,105.06414795)
\lineto(560.87857056,105.06414795)
\curveto(560.92544372,104.32195817)(561.14809975,103.75555248)(561.54653931,103.3649292)
\curveto(561.9488802,102.97820951)(562.50747339,102.78485033)(563.22232056,102.78485107)
\curveto(563.63637851,102.78485033)(564.03676874,102.83563153)(564.42349243,102.93719482)
\curveto(564.81411171,103.03875633)(565.20083007,103.19109992)(565.58364868,103.39422607)
\lineto(565.58364868,102.37469482)
\curveto(565.19692383,102.21063215)(564.80043985,102.08563228)(564.39419556,101.99969482)
\curveto(563.98794066,101.91375745)(563.5758317,101.87078874)(563.15786743,101.87078857)
\curveto(562.11098941,101.87078874)(561.28091212,102.17547594)(560.66763306,102.78485107)
\curveto(560.05825709,103.39422472)(559.75356989,104.21844265)(559.75357056,105.25750732)
\curveto(559.75356989,106.33172178)(560.04263211,107.18328343)(560.62075806,107.81219482)
\curveto(561.2027872,108.44500092)(561.98598954,108.76140685)(562.97036743,108.76141357)
\curveto(563.85317517,108.76140685)(564.5504401,108.47625089)(565.06216431,107.90594482)
\curveto(565.57778282,107.33953328)(565.83559506,106.56804967)(565.83560181,105.5914917)
\moveto(564.75747681,105.90789795)
\curveto(564.74965865,106.49773724)(564.58364319,106.9684399)(564.25942993,107.32000732)
\curveto(563.93911258,107.67156419)(563.51333176,107.84734527)(562.98208618,107.84735107)
\curveto(562.38052039,107.84734527)(561.898099,107.67742356)(561.53482056,107.33758545)
\curveto(561.17544347,106.99773674)(560.96841243,106.5192216)(560.91372681,105.90203857)
\lineto(564.75747681,105.90789795)
}
}
{
\newrgbcolor{curcolor}{1 0 0}
\pscustom[linewidth=1.02900004,linecolor=curcolor,strokeopacity=0.78431374,linestyle=dashed,dash=8.23200035 1.02900004]
{
\newpath
\moveto(260.51455,595.51455262)
\lineto(489.45549,595.51455262)
}
}
{
\newrgbcolor{curcolor}{0 0 0}
\pscustom[linestyle=none,fillstyle=solid,fillcolor=curcolor]
{
\newpath
\moveto(414.51016235,608.10876465)
\lineto(415.65274048,608.10876465)
\lineto(417.70352173,602.60095215)
\lineto(419.75430298,608.10876465)
\lineto(420.8968811,608.10876465)
\lineto(418.4359436,601.54626465)
\lineto(416.97109985,601.54626465)
\lineto(414.51016235,608.10876465)
}
}
{
\newrgbcolor{curcolor}{0 0 0}
\pscustom[linestyle=none,fillstyle=solid,fillcolor=curcolor]
{
\newpath
\moveto(422.38516235,608.10876465)
\lineto(423.46328735,608.10876465)
\lineto(423.46328735,601.54626465)
\lineto(422.38516235,601.54626465)
\lineto(422.38516235,608.10876465)
\moveto(422.38516235,610.66345215)
\lineto(423.46328735,610.66345215)
\lineto(423.46328735,609.29821777)
\lineto(422.38516235,609.29821777)
\lineto(422.38516235,610.66345215)
}
}
{
\newrgbcolor{curcolor}{0 0 0}
\pscustom[linestyle=none,fillstyle=solid,fillcolor=curcolor]
{
\newpath
\moveto(429.51602173,607.10095215)
\curveto(429.39492317,607.17125902)(429.2621108,607.22204022)(429.11758423,607.2532959)
\curveto(428.97695483,607.28844641)(428.82070499,607.30602451)(428.64883423,607.30603027)
\curveto(428.03945577,607.30602451)(427.57070624,607.10680596)(427.24258423,606.70837402)
\curveto(426.91836314,606.31383801)(426.75625393,605.7454792)(426.7562561,605.0032959)
\lineto(426.7562561,601.54626465)
\lineto(425.67227173,601.54626465)
\lineto(425.67227173,608.10876465)
\lineto(426.7562561,608.10876465)
\lineto(426.7562561,607.0892334)
\curveto(426.9828162,607.48766496)(427.27773778,607.78258654)(427.64102173,607.97399902)
\curveto(428.00429956,608.1693049)(428.44570537,608.26696105)(428.96524048,608.26696777)
\curveto(429.03945477,608.26696105)(429.12148594,608.26110168)(429.21133423,608.24938965)
\curveto(429.30117326,608.24157045)(429.40078254,608.22789859)(429.51016235,608.20837402)
\lineto(429.51602173,607.10095215)
}
}
{
\newrgbcolor{curcolor}{0 0 0}
\pscustom[linestyle=none,fillstyle=solid,fillcolor=curcolor]
{
\newpath
\moveto(431.7250061,609.9720459)
\lineto(431.7250061,608.10876465)
\lineto(433.94570923,608.10876465)
\lineto(433.94570923,607.27087402)
\lineto(431.7250061,607.27087402)
\lineto(431.7250061,603.70837402)
\curveto(431.72500391,603.17321615)(431.79726946,602.82946649)(431.94180298,602.67712402)
\curveto(432.09023792,602.52477929)(432.38906574,602.4486075)(432.83828735,602.4486084)
\lineto(433.94570923,602.4486084)
\lineto(433.94570923,601.54626465)
\lineto(432.83828735,601.54626465)
\curveto(432.00625362,601.54626465)(431.43203545,601.70056137)(431.1156311,602.00915527)
\curveto(430.79922358,602.3216545)(430.64102062,602.88806018)(430.64102173,603.70837402)
\lineto(430.64102173,607.27087402)
\lineto(429.8500061,607.27087402)
\lineto(429.8500061,608.10876465)
\lineto(430.64102173,608.10876465)
\lineto(430.64102173,609.9720459)
\lineto(431.7250061,609.9720459)
}
}
{
\newrgbcolor{curcolor}{0 0 0}
\pscustom[linestyle=none,fillstyle=solid,fillcolor=curcolor]
{
\newpath
\moveto(435.25820923,604.1361084)
\lineto(435.25820923,608.10876465)
\lineto(436.33633423,608.10876465)
\lineto(436.33633423,604.17712402)
\curveto(436.33633213,603.55602826)(436.45742576,603.08923186)(436.69961548,602.7767334)
\curveto(436.94180028,602.46813873)(437.30508116,602.31384201)(437.78945923,602.31384277)
\curveto(438.37148635,602.31384201)(438.83047026,602.4993887)(439.16641235,602.8704834)
\curveto(439.50625084,603.24157545)(439.67617254,603.74743432)(439.67617798,604.38806152)
\lineto(439.67617798,608.10876465)
\lineto(440.75430298,608.10876465)
\lineto(440.75430298,601.54626465)
\lineto(439.67617798,601.54626465)
\lineto(439.67617798,602.55407715)
\curveto(439.41445405,602.15563904)(439.10976686,601.85876434)(438.76211548,601.66345215)
\curveto(438.4183613,601.47204597)(438.01797107,601.37634294)(437.5609436,601.37634277)
\curveto(436.80703479,601.37634294)(436.23476973,601.61071771)(435.84414673,602.07946777)
\curveto(435.45352051,602.54821677)(435.25820821,603.23376296)(435.25820923,604.1361084)
\moveto(437.97109985,608.26696777)
\lineto(437.97109985,608.26696777)
}
}
{
\newrgbcolor{curcolor}{0 0 0}
\pscustom[linestyle=none,fillstyle=solid,fillcolor=curcolor]
{
\newpath
\moveto(448.6000061,605.0970459)
\lineto(448.6000061,604.56970215)
\lineto(443.64297485,604.56970215)
\curveto(443.68984802,603.82751237)(443.91250405,603.26110668)(444.3109436,602.8704834)
\curveto(444.7132845,602.48376371)(445.27187769,602.29040453)(445.98672485,602.29040527)
\curveto(446.40078281,602.29040453)(446.80117303,602.34118573)(447.18789673,602.44274902)
\curveto(447.57851601,602.54431053)(447.96523437,602.69665412)(448.34805298,602.89978027)
\lineto(448.34805298,601.88024902)
\curveto(447.96132812,601.71618635)(447.56484414,601.59118648)(447.15859985,601.50524902)
\curveto(446.75234496,601.41931165)(446.34023599,601.37634294)(445.92227173,601.37634277)
\curveto(444.87539371,601.37634294)(444.04531641,601.68103014)(443.43203735,602.29040527)
\curveto(442.82266139,602.89977892)(442.51797419,603.72399685)(442.51797485,604.76306152)
\curveto(442.51797419,605.83727598)(442.8070364,606.68883763)(443.38516235,607.31774902)
\curveto(443.96719149,607.95055512)(444.75039383,608.26696105)(445.73477173,608.26696777)
\curveto(446.61757947,608.26696105)(447.31484439,607.98180509)(447.8265686,607.41149902)
\curveto(448.34218712,606.84508747)(448.59999936,606.07360387)(448.6000061,605.0970459)
\moveto(447.5218811,605.41345215)
\curveto(447.51406295,606.00329144)(447.34804749,606.4739941)(447.02383423,606.82556152)
\curveto(446.70351688,607.17711839)(446.27773606,607.35289947)(445.74649048,607.35290527)
\curveto(445.14492469,607.35289947)(444.6625033,607.18297776)(444.29922485,606.84313965)
\curveto(443.93984777,606.50329094)(443.73281673,606.02477579)(443.6781311,605.40759277)
\lineto(447.5218811,605.41345215)
}
}
{
\newrgbcolor{curcolor}{0 0 0}
\pscustom[linestyle=none,fillstyle=solid,fillcolor=curcolor]
{
\newpath
\moveto(450.36953735,610.66345215)
\lineto(451.44766235,610.66345215)
\lineto(451.44766235,601.54626465)
\lineto(450.36953735,601.54626465)
\lineto(450.36953735,610.66345215)
}
}
{
\newrgbcolor{curcolor}{0 0 0}
\pscustom[linestyle=none,fillstyle=solid,fillcolor=curcolor]
{
\newpath
\moveto(453.69766235,610.66345215)
\lineto(454.77578735,610.66345215)
\lineto(454.77578735,601.54626465)
\lineto(453.69766235,601.54626465)
\lineto(453.69766235,610.66345215)
}
}
{
\newrgbcolor{curcolor}{0 0 0}
\pscustom[linestyle=none,fillstyle=solid,fillcolor=curcolor]
{
\newpath
\moveto(462.6390686,605.0970459)
\lineto(462.6390686,604.56970215)
\lineto(457.68203735,604.56970215)
\curveto(457.72891052,603.82751237)(457.95156655,603.26110668)(458.3500061,602.8704834)
\curveto(458.752347,602.48376371)(459.31094019,602.29040453)(460.02578735,602.29040527)
\curveto(460.43984531,602.29040453)(460.84023553,602.34118573)(461.22695923,602.44274902)
\curveto(461.61757851,602.54431053)(462.00429687,602.69665412)(462.38711548,602.89978027)
\lineto(462.38711548,601.88024902)
\curveto(462.00039062,601.71618635)(461.60390664,601.59118648)(461.19766235,601.50524902)
\curveto(460.79140746,601.41931165)(460.37929849,601.37634294)(459.96133423,601.37634277)
\curveto(458.91445621,601.37634294)(458.08437891,601.68103014)(457.47109985,602.29040527)
\curveto(456.86172389,602.89977892)(456.55703669,603.72399685)(456.55703735,604.76306152)
\curveto(456.55703669,605.83727598)(456.8460989,606.68883763)(457.42422485,607.31774902)
\curveto(458.00625399,607.95055512)(458.78945633,608.26696105)(459.77383423,608.26696777)
\curveto(460.65664197,608.26696105)(461.35390689,607.98180509)(461.8656311,607.41149902)
\curveto(462.38124962,606.84508747)(462.63906186,606.07360387)(462.6390686,605.0970459)
\moveto(461.5609436,605.41345215)
\curveto(461.55312545,606.00329144)(461.38710999,606.4739941)(461.06289673,606.82556152)
\curveto(460.74257938,607.17711839)(460.31679856,607.35289947)(459.78555298,607.35290527)
\curveto(459.18398719,607.35289947)(458.7015658,607.18297776)(458.33828735,606.84313965)
\curveto(457.97891027,606.50329094)(457.77187923,606.02477579)(457.7171936,605.40759277)
\lineto(461.5609436,605.41345215)
}
}
{
\newrgbcolor{curcolor}{0 0 0}
\pscustom[linestyle=none,fillstyle=solid,fillcolor=curcolor]
{
}
}
{
\newrgbcolor{curcolor}{0 0 0}
\pscustom[linestyle=none,fillstyle=solid,fillcolor=curcolor]
{
\newpath
\moveto(469.4593811,605.72399902)
\lineto(469.4593811,602.5189209)
\lineto(471.3578186,602.5189209)
\curveto(471.99453246,602.51891993)(472.46523511,602.64977917)(472.76992798,602.91149902)
\curveto(473.07851575,603.17712239)(473.23281247,603.58141886)(473.2328186,604.12438965)
\curveto(473.23281247,604.67126152)(473.07851575,605.07360487)(472.76992798,605.3314209)
\curveto(472.46523511,605.5931356)(471.99453246,605.72399485)(471.3578186,605.72399902)
\lineto(469.4593811,605.72399902)
\moveto(469.4593811,609.32165527)
\lineto(469.4593811,606.68493652)
\lineto(471.21133423,606.68493652)
\curveto(471.78945454,606.68493138)(472.21914161,606.79235315)(472.50039673,607.00720215)
\curveto(472.78554729,607.22594647)(472.92812527,607.55797739)(472.9281311,608.0032959)
\curveto(472.92812527,608.44469525)(472.78554729,608.77477304)(472.50039673,608.99353027)
\curveto(472.21914161,609.21227261)(471.78945454,609.3216475)(471.21133423,609.32165527)
\lineto(469.4593811,609.32165527)
\moveto(468.27578735,610.29431152)
\lineto(471.29922485,610.29431152)
\curveto(472.2015635,610.29430278)(472.8968753,610.10680296)(473.38516235,609.73181152)
\curveto(473.87343683,609.35680371)(474.11757721,608.82360112)(474.11758423,608.13220215)
\curveto(474.11757721,607.59703985)(473.99257733,607.17125902)(473.74258423,606.8548584)
\curveto(473.49257783,606.53844716)(473.1253907,606.34118173)(472.64102173,606.26306152)
\curveto(473.22304685,606.13805693)(473.67421828,605.87633844)(473.99453735,605.47790527)
\curveto(474.31874888,605.08337049)(474.4808581,604.58923036)(474.48086548,603.9954834)
\curveto(474.4808581,603.21423173)(474.21523336,602.61071671)(473.68399048,602.18493652)
\curveto(473.15273442,601.75915506)(472.3968758,601.54626465)(471.41641235,601.54626465)
\lineto(468.27578735,601.54626465)
\lineto(468.27578735,610.29431152)
}
}
{
\newrgbcolor{curcolor}{0 0 0}
\pscustom[linestyle=none,fillstyle=solid,fillcolor=curcolor]
{
\newpath
\moveto(476.45547485,608.10876465)
\lineto(477.53359985,608.10876465)
\lineto(477.53359985,601.54626465)
\lineto(476.45547485,601.54626465)
\lineto(476.45547485,608.10876465)
\moveto(476.45547485,610.66345215)
\lineto(477.53359985,610.66345215)
\lineto(477.53359985,609.29821777)
\lineto(476.45547485,609.29821777)
\lineto(476.45547485,610.66345215)
}
}
{
\newrgbcolor{curcolor}{0 0 0}
\pscustom[linestyle=none,fillstyle=solid,fillcolor=curcolor]
{
\newpath
\moveto(479.78359985,610.66345215)
\lineto(480.86172485,610.66345215)
\lineto(480.86172485,601.54626465)
\lineto(479.78359985,601.54626465)
\lineto(479.78359985,610.66345215)
}
}
{
\newrgbcolor{curcolor}{0 0 0}
\pscustom[linestyle=none,fillstyle=solid,fillcolor=curcolor]
{
\newpath
\moveto(487.43008423,607.1126709)
\lineto(487.43008423,610.66345215)
\lineto(488.50820923,610.66345215)
\lineto(488.50820923,601.54626465)
\lineto(487.43008423,601.54626465)
\lineto(487.43008423,602.53063965)
\curveto(487.20351651,602.14001405)(486.91640742,601.84899872)(486.5687561,601.65759277)
\curveto(486.22500186,601.47009285)(485.81093977,601.37634294)(485.3265686,601.37634277)
\curveto(484.5335973,601.37634294)(483.88711357,601.69274888)(483.38711548,602.32556152)
\curveto(482.89102082,602.95837261)(482.64297419,603.79040303)(482.64297485,604.82165527)
\curveto(482.64297419,605.85290097)(482.89102082,606.68493138)(483.38711548,607.31774902)
\curveto(483.88711357,607.95055512)(484.5335973,608.26696105)(485.3265686,608.26696777)
\curveto(485.81093977,608.26696105)(486.22500186,608.17125802)(486.5687561,607.9798584)
\curveto(486.91640742,607.79235215)(487.20351651,607.50328994)(487.43008423,607.1126709)
\moveto(483.7562561,604.82165527)
\curveto(483.75625433,604.02868404)(483.91836354,603.40563779)(484.24258423,602.95251465)
\curveto(484.57070664,602.50329494)(485.01992494,602.27868579)(485.59024048,602.27868652)
\curveto(486.1605488,602.27868579)(486.6097671,602.50329494)(486.93789673,602.95251465)
\curveto(487.26601644,603.40563779)(487.43007878,604.02868404)(487.43008423,604.82165527)
\curveto(487.43007878,605.61461996)(487.26601644,606.23571308)(486.93789673,606.68493652)
\curveto(486.6097671,607.13805593)(486.1605488,607.36461821)(485.59024048,607.36462402)
\curveto(485.01992494,607.36461821)(484.57070664,607.13805593)(484.24258423,606.68493652)
\curveto(483.91836354,606.23571308)(483.75625433,605.61461996)(483.7562561,604.82165527)
}
}
{
\newrgbcolor{curcolor}{0 0 0}
\pscustom[linestyle=none,fillstyle=solid,fillcolor=curcolor]
{
\newpath
\moveto(496.3421936,605.0970459)
\lineto(496.3421936,604.56970215)
\lineto(491.38516235,604.56970215)
\curveto(491.43203552,603.82751237)(491.65469155,603.26110668)(492.0531311,602.8704834)
\curveto(492.455472,602.48376371)(493.01406519,602.29040453)(493.72891235,602.29040527)
\curveto(494.14297031,602.29040453)(494.54336053,602.34118573)(494.93008423,602.44274902)
\curveto(495.32070351,602.54431053)(495.70742187,602.69665412)(496.09024048,602.89978027)
\lineto(496.09024048,601.88024902)
\curveto(495.70351562,601.71618635)(495.30703164,601.59118648)(494.90078735,601.50524902)
\curveto(494.49453246,601.41931165)(494.08242349,601.37634294)(493.66445923,601.37634277)
\curveto(492.61758121,601.37634294)(491.78750391,601.68103014)(491.17422485,602.29040527)
\curveto(490.56484889,602.89977892)(490.26016169,603.72399685)(490.26016235,604.76306152)
\curveto(490.26016169,605.83727598)(490.5492239,606.68883763)(491.12734985,607.31774902)
\curveto(491.70937899,607.95055512)(492.49258133,608.26696105)(493.47695923,608.26696777)
\curveto(494.35976697,608.26696105)(495.05703189,607.98180509)(495.5687561,607.41149902)
\curveto(496.08437462,606.84508747)(496.34218686,606.07360387)(496.3421936,605.0970459)
\moveto(495.2640686,605.41345215)
\curveto(495.25625045,606.00329144)(495.09023499,606.4739941)(494.76602173,606.82556152)
\curveto(494.44570438,607.17711839)(494.01992356,607.35289947)(493.48867798,607.35290527)
\curveto(492.88711219,607.35289947)(492.4046908,607.18297776)(492.04141235,606.84313965)
\curveto(491.68203527,606.50329094)(491.47500423,606.02477579)(491.4203186,605.40759277)
\lineto(495.2640686,605.41345215)
}
}
{
\newrgbcolor{curcolor}{0 0 0}
\pscustom[linestyle=none,fillstyle=solid,fillcolor=curcolor]
{
\newpath
\moveto(502.82266235,604.82165527)
\curveto(502.82265651,605.61461996)(502.65859418,606.23571308)(502.33047485,606.68493652)
\curveto(502.00625108,607.13805593)(501.5589859,607.36461821)(500.98867798,607.36462402)
\curveto(500.41836204,607.36461821)(499.96914374,607.13805593)(499.64102173,606.68493652)
\curveto(499.31680064,606.23571308)(499.15469143,605.61461996)(499.1546936,604.82165527)
\curveto(499.15469143,604.02868404)(499.31680064,603.40563779)(499.64102173,602.95251465)
\curveto(499.96914374,602.50329494)(500.41836204,602.27868579)(500.98867798,602.27868652)
\curveto(501.5589859,602.27868579)(502.00625108,602.50329494)(502.33047485,602.95251465)
\curveto(502.65859418,603.40563779)(502.82265651,604.02868404)(502.82266235,604.82165527)
\moveto(499.1546936,607.1126709)
\curveto(499.3812537,607.50328994)(499.66640967,607.79235215)(500.01016235,607.9798584)
\curveto(500.35781523,608.17125802)(500.77187731,608.26696105)(501.25234985,608.26696777)
\curveto(502.04921979,608.26696105)(502.69570351,607.95055512)(503.19180298,607.31774902)
\curveto(503.69179627,606.68493138)(503.94179602,605.85290097)(503.94180298,604.82165527)
\curveto(503.94179602,603.79040303)(503.69179627,602.95837261)(503.19180298,602.32556152)
\curveto(502.69570351,601.69274888)(502.04921979,601.37634294)(501.25234985,601.37634277)
\curveto(500.77187731,601.37634294)(500.35781523,601.47009285)(500.01016235,601.65759277)
\curveto(499.66640967,601.84899872)(499.3812537,602.14001405)(499.1546936,602.53063965)
\lineto(499.1546936,601.54626465)
\lineto(498.07070923,601.54626465)
\lineto(498.07070923,610.66345215)
\lineto(499.1546936,610.66345215)
\lineto(499.1546936,607.1126709)
}
}
{
\newrgbcolor{curcolor}{0 0 0}
\pscustom[linestyle=none,fillstyle=solid,fillcolor=curcolor]
{
\newpath
\moveto(511.3421936,605.0970459)
\lineto(511.3421936,604.56970215)
\lineto(506.38516235,604.56970215)
\curveto(506.43203552,603.82751237)(506.65469155,603.26110668)(507.0531311,602.8704834)
\curveto(507.455472,602.48376371)(508.01406519,602.29040453)(508.72891235,602.29040527)
\curveto(509.14297031,602.29040453)(509.54336053,602.34118573)(509.93008423,602.44274902)
\curveto(510.32070351,602.54431053)(510.70742187,602.69665412)(511.09024048,602.89978027)
\lineto(511.09024048,601.88024902)
\curveto(510.70351562,601.71618635)(510.30703164,601.59118648)(509.90078735,601.50524902)
\curveto(509.49453246,601.41931165)(509.08242349,601.37634294)(508.66445923,601.37634277)
\curveto(507.61758121,601.37634294)(506.78750391,601.68103014)(506.17422485,602.29040527)
\curveto(505.56484889,602.89977892)(505.26016169,603.72399685)(505.26016235,604.76306152)
\curveto(505.26016169,605.83727598)(505.5492239,606.68883763)(506.12734985,607.31774902)
\curveto(506.70937899,607.95055512)(507.49258133,608.26696105)(508.47695923,608.26696777)
\curveto(509.35976697,608.26696105)(510.05703189,607.98180509)(510.5687561,607.41149902)
\curveto(511.08437462,606.84508747)(511.34218686,606.07360387)(511.3421936,605.0970459)
\moveto(510.2640686,605.41345215)
\curveto(510.25625045,606.00329144)(510.09023499,606.4739941)(509.76602173,606.82556152)
\curveto(509.44570438,607.17711839)(509.01992356,607.35289947)(508.48867798,607.35290527)
\curveto(507.88711219,607.35289947)(507.4046908,607.18297776)(507.04141235,606.84313965)
\curveto(506.68203527,606.50329094)(506.47500423,606.02477579)(506.4203186,605.40759277)
\lineto(510.2640686,605.41345215)
}
}
{
\newrgbcolor{curcolor}{0 0 0}
\pscustom[linestyle=none,fillstyle=solid,fillcolor=curcolor]
{
\newpath
\moveto(518.56680298,605.50720215)
\lineto(518.56680298,601.54626465)
\lineto(517.48867798,601.54626465)
\lineto(517.48867798,605.4720459)
\curveto(517.48867247,606.0931351)(517.36757884,606.55797839)(517.12539673,606.86657715)
\curveto(516.88320433,607.17516527)(516.51992344,607.32946199)(516.03555298,607.32946777)
\curveto(515.45351826,607.32946199)(514.99453434,607.1439153)(514.65859985,606.77282715)
\curveto(514.32266001,606.40172854)(514.15469143,605.89586967)(514.1546936,605.25524902)
\lineto(514.1546936,601.54626465)
\lineto(513.07070923,601.54626465)
\lineto(513.07070923,608.10876465)
\lineto(514.1546936,608.10876465)
\lineto(514.1546936,607.0892334)
\curveto(514.41250367,607.48375871)(514.71523774,607.77868029)(515.06289673,607.97399902)
\curveto(515.41445579,608.1693049)(515.81875227,608.26696105)(516.27578735,608.26696777)
\curveto(517.02968855,608.26696105)(517.60000048,608.03258629)(517.98672485,607.56384277)
\curveto(518.37343721,607.09899347)(518.56679639,606.41344728)(518.56680298,605.50720215)
}
}
{
\newrgbcolor{curcolor}{0 0 0}
\pscustom[linestyle=none,fillstyle=solid,fillcolor=curcolor]
{
\newpath
\moveto(526.3421936,605.0970459)
\lineto(526.3421936,604.56970215)
\lineto(521.38516235,604.56970215)
\curveto(521.43203552,603.82751237)(521.65469155,603.26110668)(522.0531311,602.8704834)
\curveto(522.455472,602.48376371)(523.01406519,602.29040453)(523.72891235,602.29040527)
\curveto(524.14297031,602.29040453)(524.54336053,602.34118573)(524.93008423,602.44274902)
\curveto(525.32070351,602.54431053)(525.70742187,602.69665412)(526.09024048,602.89978027)
\lineto(526.09024048,601.88024902)
\curveto(525.70351562,601.71618635)(525.30703164,601.59118648)(524.90078735,601.50524902)
\curveto(524.49453246,601.41931165)(524.08242349,601.37634294)(523.66445923,601.37634277)
\curveto(522.61758121,601.37634294)(521.78750391,601.68103014)(521.17422485,602.29040527)
\curveto(520.56484889,602.89977892)(520.26016169,603.72399685)(520.26016235,604.76306152)
\curveto(520.26016169,605.83727598)(520.5492239,606.68883763)(521.12734985,607.31774902)
\curveto(521.70937899,607.95055512)(522.49258133,608.26696105)(523.47695923,608.26696777)
\curveto(524.35976697,608.26696105)(525.05703189,607.98180509)(525.5687561,607.41149902)
\curveto(526.08437462,606.84508747)(526.34218686,606.07360387)(526.3421936,605.0970459)
\moveto(525.2640686,605.41345215)
\curveto(525.25625045,606.00329144)(525.09023499,606.4739941)(524.76602173,606.82556152)
\curveto(524.44570438,607.17711839)(524.01992356,607.35289947)(523.48867798,607.35290527)
\curveto(522.88711219,607.35289947)(522.4046908,607.18297776)(522.04141235,606.84313965)
\curveto(521.68203527,606.50329094)(521.47500423,606.02477579)(521.4203186,605.40759277)
\lineto(525.2640686,605.41345215)
}
}
{
\newrgbcolor{curcolor}{0 0 0}
\pscustom[linestyle=none,fillstyle=solid,fillcolor=curcolor]
{
\newpath
\moveto(247.06952872,907.85934353)
\lineto(243.51654756,913.27783032)
\lineto(239.61493276,908.22370605)
\lineto(242.7699168,913.99257223)
\lineto(237.59732039,916.92698964)
\lineto(243.10018891,915.07385822)
\lineto(243.80496332,921.94155219)
\lineto(244.05093907,915.02738781)
\lineto(249.65911002,916.3374387)
\lineto(244.30826287,913.91738152)
\closepath
}
}
{
\newrgbcolor{curcolor}{0 0 0}
\pscustom[linewidth=0.49342826,linecolor=curcolor,strokeopacity=0.78431374,linestyle=dashed,dash=8.23200035 1.02900004]
{
\newpath
\moveto(247.06952872,907.85934353)
\lineto(243.51654756,913.27783032)
\lineto(239.61493276,908.22370605)
\lineto(242.7699168,913.99257223)
\lineto(237.59732039,916.92698964)
\lineto(243.10018891,915.07385822)
\lineto(243.80496332,921.94155219)
\lineto(244.05093907,915.02738781)
\lineto(249.65911002,916.3374387)
\lineto(244.30826287,913.91738152)
\closepath
}
}
{
\newrgbcolor{curcolor}{0 0 0}
\pscustom[linewidth=1.00349939,linecolor=curcolor]
{
\newpath
\moveto(243.44851,915.19645262)
\lineto(433.22742,105.18388262)
}
}
{
\newrgbcolor{curcolor}{0 0 0}
\pscustom[linewidth=1,linecolor=curcolor]
{
\newpath
\moveto(243.44676,915.19820262)
\lineto(374.76659,915.19820262)
}
}
{
\newrgbcolor{curcolor}{0 0 0}
\pscustom[linestyle=none,fillstyle=solid,fillcolor=curcolor]
{
\newpath
\moveto(253.44676,915.19820262)
\lineto(257.44676,919.19820262)
\lineto(243.44676,915.19820262)
\lineto(257.44676,911.19820262)
\lineto(253.44676,915.19820262)
\closepath
}
}
{
\newrgbcolor{curcolor}{0 0 0}
\pscustom[linewidth=1,linecolor=curcolor]
{
\newpath
\moveto(253.44676,915.19820262)
\lineto(257.44676,919.19820262)
\lineto(243.44676,915.19820262)
\lineto(257.44676,911.19820262)
\lineto(253.44676,915.19820262)
\closepath
}
}
{
\newrgbcolor{curcolor}{0 0 0}
\pscustom[linewidth=1,linecolor=curcolor]
{
\newpath
\moveto(375.52421,595.42179262)
\lineto(318.19805,595.42179262)
}
}
{
\newrgbcolor{curcolor}{0 0 0}
\pscustom[linestyle=none,fillstyle=solid,fillcolor=curcolor]
{
\newpath
\moveto(328.19805,595.42179262)
\lineto(332.19805,599.42179262)
\lineto(318.19805,595.42179262)
\lineto(332.19805,591.42179262)
\lineto(328.19805,595.42179262)
\closepath
}
}
{
\newrgbcolor{curcolor}{0 0 0}
\pscustom[linewidth=1,linecolor=curcolor]
{
\newpath
\moveto(328.19805,595.42179262)
\lineto(332.19805,599.42179262)
\lineto(318.19805,595.42179262)
\lineto(332.19805,591.42179262)
\lineto(328.19805,595.42179262)
\closepath
}
}
{
\newrgbcolor{curcolor}{0 0 0}
\pscustom[linewidth=1,linecolor=curcolor]
{
\newpath
\moveto(375.39794,105.56094262)
\lineto(432.97664,105.56094262)
}
}
{
\newrgbcolor{curcolor}{0 0 0}
\pscustom[linestyle=none,fillstyle=solid,fillcolor=curcolor]
{
\newpath
\moveto(422.97664,105.56094262)
\lineto(418.97664,101.56094262)
\lineto(432.97664,105.56094262)
\lineto(418.97664,109.56094262)
\lineto(422.97664,105.56094262)
\closepath
}
}
{
\newrgbcolor{curcolor}{0 0 0}
\pscustom[linewidth=1,linecolor=curcolor]
{
\newpath
\moveto(422.97664,105.56094262)
\lineto(418.97664,101.56094262)
\lineto(432.97664,105.56094262)
\lineto(418.97664,109.56094262)
\lineto(422.97664,105.56094262)
\closepath
}
}
{
\newrgbcolor{curcolor}{0 0 0}
\pscustom[linestyle=none,fillstyle=solid,fillcolor=curcolor]
{
\newpath
\moveto(213.48321533,927.01420593)
\lineto(213.48321533,923.72709656)
\lineto(214.97149658,923.72709656)
\curveto(215.52227343,923.72709207)(215.94805426,923.86967005)(216.24884033,924.15483093)
\curveto(216.54961615,924.43998198)(216.70000663,924.84623158)(216.70001221,925.37358093)
\curveto(216.70000663,925.89701177)(216.54961615,926.30130825)(216.24884033,926.58647156)
\curveto(215.94805426,926.87162017)(215.52227343,927.01419816)(214.97149658,927.01420593)
\lineto(213.48321533,927.01420593)
\moveto(212.29962158,927.98686218)
\lineto(214.97149658,927.98686218)
\curveto(215.9519605,927.98685343)(216.69219414,927.76419741)(217.19219971,927.31889343)
\curveto(217.69609938,926.87747954)(217.94805226,926.22904269)(217.94805908,925.37358093)
\curveto(217.94805226,924.51029441)(217.69609938,923.85795131)(217.19219971,923.41654968)
\curveto(216.69219414,922.9751397)(215.9519605,922.75443679)(214.97149658,922.75444031)
\lineto(213.48321533,922.75444031)
\lineto(213.48321533,919.23881531)
\lineto(212.29962158,919.23881531)
\lineto(212.29962158,927.98686218)
}
}
{
\newrgbcolor{curcolor}{0 0 0}
\pscustom[linestyle=none,fillstyle=solid,fillcolor=curcolor]
{
\newpath
\moveto(451.87921143,92.62817383)
\lineto(451.87921143,89.34106445)
\lineto(453.36749268,89.34106445)
\curveto(453.91826953,89.34105996)(454.34405035,89.48363795)(454.64483643,89.76879883)
\curveto(454.94561225,90.05394988)(455.09600272,90.46019947)(455.0960083,90.98754883)
\curveto(455.09600272,91.51097967)(454.94561225,91.91527614)(454.64483643,92.20043945)
\curveto(454.34405035,92.48558807)(453.91826953,92.62816605)(453.36749268,92.62817383)
\lineto(451.87921143,92.62817383)
\moveto(450.69561768,93.60083008)
\lineto(453.36749268,93.60083008)
\curveto(454.3479566,93.60082133)(455.08819023,93.3781653)(455.5881958,92.93286133)
\curveto(456.09209548,92.49144744)(456.34404835,91.84301059)(456.34405518,90.98754883)
\curveto(456.34404835,90.12426231)(456.09209548,89.47191921)(455.5881958,89.03051758)
\curveto(455.08819023,88.58910759)(454.3479566,88.36840469)(453.36749268,88.3684082)
\lineto(451.87921143,88.3684082)
\lineto(451.87921143,84.8527832)
\lineto(450.69561768,84.8527832)
\lineto(450.69561768,93.60083008)
}
}
{
\newrgbcolor{curcolor}{0 0 0}
\pscustom[linestyle=none,fillstyle=solid,fillcolor=curcolor]
{
\newpath
\moveto(458.91046143,93.60083008)
\lineto(458.91046143,90.34887695)
\lineto(457.91436768,90.34887695)
\lineto(457.91436768,93.60083008)
\lineto(458.91046143,93.60083008)
}
}
{
\newrgbcolor{curcolor}{0 0 0}
\pscustom[linestyle=none,fillstyle=solid,fillcolor=curcolor]
{
\newpath
\moveto(292.71847534,589.20394897)
\lineto(292.71847534,585.9168396)
\lineto(294.20675659,585.9168396)
\curveto(294.75753344,585.91683511)(295.18331427,586.05941309)(295.48410034,586.34457397)
\curveto(295.78487616,586.62972502)(295.93526664,587.03597462)(295.93527222,587.56332397)
\curveto(295.93526664,588.08675482)(295.78487616,588.49105129)(295.48410034,588.7762146)
\curveto(295.18331427,589.06136322)(294.75753344,589.2039412)(294.20675659,589.20394897)
\lineto(292.71847534,589.20394897)
\moveto(291.53488159,590.17660522)
\lineto(294.20675659,590.17660522)
\curveto(295.18722051,590.17659648)(295.92745415,589.95394045)(296.42745972,589.50863647)
\curveto(296.93135939,589.06722259)(297.18331227,588.41878573)(297.18331909,587.56332397)
\curveto(297.18331227,586.70003745)(296.93135939,586.04769436)(296.42745972,585.60629272)
\curveto(295.92745415,585.16488274)(295.18722051,584.94417983)(294.20675659,584.94418335)
\lineto(292.71847534,584.94418335)
\lineto(292.71847534,581.42855835)
\lineto(291.53488159,581.42855835)
\lineto(291.53488159,590.17660522)
}
}
{
\newrgbcolor{curcolor}{0 0 0}
\pscustom[linestyle=none,fillstyle=solid,fillcolor=curcolor]
{
\newpath
\moveto(303.24191284,588.73519897)
\lineto(301.13839722,587.59848022)
\lineto(303.24191284,586.4559021)
\lineto(302.90206909,585.88168335)
\lineto(300.93331909,587.07113647)
\lineto(300.93331909,584.8621521)
\lineto(300.26535034,584.8621521)
\lineto(300.26535034,587.07113647)
\lineto(298.29660034,585.88168335)
\lineto(297.95675659,586.4559021)
\lineto(300.06027222,587.59848022)
\lineto(297.95675659,588.73519897)
\lineto(298.29660034,589.3152771)
\lineto(300.26535034,588.12582397)
\lineto(300.26535034,590.33480835)
\lineto(300.93331909,590.33480835)
\lineto(300.93331909,588.12582397)
\lineto(302.90206909,589.3152771)
\lineto(303.24191284,588.73519897)
}
}
{
\newrgbcolor{curcolor}{0 0 0}
\pscustom[linewidth=1.16166615,linecolor=curcolor,linestyle=dashed,dash=1.16166615 4.64666462]
{
\newpath
\moveto(375.58083,915.19821262)
\lineto(624.23137,915.19821262)
}
}
{
\newrgbcolor{curcolor}{0 0 0}
\pscustom[linewidth=1.06639314,linecolor=curcolor,linestyle=dashed,dash=1.06639314 4.26557255]
{
\newpath
\moveto(375.5332,349.28571262)
\lineto(624.396,349.28571262)
}
}
{
\newrgbcolor{curcolor}{0 0 0}
\pscustom[linewidth=1.00032091,linecolor=curcolor]
{
\newpath
\moveto(450.71429,349.84248262)
\lineto(450.71429,595.42179262)
}
}
{
\newrgbcolor{curcolor}{0 0 0}
\pscustom[linestyle=none,fillstyle=solid,fillcolor=curcolor]
{
\newpath
\moveto(450.71429,359.84569173)
\lineto(446.71300635,363.84697538)
\lineto(450.71429,349.84248262)
\lineto(454.71557365,363.84697538)
\lineto(450.71429,359.84569173)
\closepath
}
}
{
\newrgbcolor{curcolor}{0 0 0}
\pscustom[linewidth=1.00032091,linecolor=curcolor]
{
\newpath
\moveto(450.71429,359.84569173)
\lineto(446.71300635,363.84697538)
\lineto(450.71429,349.84248262)
\lineto(454.71557365,363.84697538)
\lineto(450.71429,359.84569173)
\closepath
}
}
{
\newrgbcolor{curcolor}{0 0 0}
\pscustom[linestyle=none,fillstyle=solid,fillcolor=curcolor]
{
\newpath
\moveto(450.71429,585.4185835)
\lineto(454.71557365,581.41729986)
\lineto(450.71429,595.42179262)
\lineto(446.71300635,581.41729986)
\lineto(450.71429,585.4185835)
\closepath
}
}
{
\newrgbcolor{curcolor}{0 0 0}
\pscustom[linewidth=1.00032091,linecolor=curcolor]
{
\newpath
\moveto(450.71429,585.4185835)
\lineto(454.71557365,581.41729986)
\lineto(450.71429,595.42179262)
\lineto(446.71300635,581.41729986)
\lineto(450.71429,585.4185835)
\closepath
}
}
{
\newrgbcolor{curcolor}{0 0 0}
\pscustom[linewidth=1,linecolor=curcolor]
{
\newpath
\moveto(558.57143,349.24107262)
\lineto(558.57143,915.13507262)
}
}
{
\newrgbcolor{curcolor}{0 0 0}
\pscustom[linestyle=none,fillstyle=solid,fillcolor=curcolor]
{
\newpath
\moveto(558.57143,359.24107262)
\lineto(554.57143,363.24107262)
\lineto(558.57143,349.24107262)
\lineto(562.57143,363.24107262)
\lineto(558.57143,359.24107262)
\closepath
}
}
{
\newrgbcolor{curcolor}{0 0 0}
\pscustom[linewidth=1,linecolor=curcolor]
{
\newpath
\moveto(558.57143,359.24107262)
\lineto(554.57143,363.24107262)
\lineto(558.57143,349.24107262)
\lineto(562.57143,363.24107262)
\lineto(558.57143,359.24107262)
\closepath
}
}
{
\newrgbcolor{curcolor}{0 0 0}
\pscustom[linestyle=none,fillstyle=solid,fillcolor=curcolor]
{
\newpath
\moveto(558.57143,905.13507262)
\lineto(562.57143,901.13507262)
\lineto(558.57143,915.13507262)
\lineto(554.57143,901.13507262)
\lineto(558.57143,905.13507262)
\closepath
}
}
{
\newrgbcolor{curcolor}{0 0 0}
\pscustom[linewidth=1,linecolor=curcolor]
{
\newpath
\moveto(558.57143,905.13507262)
\lineto(562.57143,901.13507262)
\lineto(558.57143,915.13507262)
\lineto(554.57143,901.13507262)
\lineto(558.57143,905.13507262)
\closepath
}
}
{
\newrgbcolor{curcolor}{0 0 0}
\pscustom[linestyle=none,fillstyle=solid,fillcolor=curcolor]
{
\newpath
\moveto(386.29995728,363.07421875)
\lineto(386.29995728,361.82617188)
\curveto(385.90151245,362.19725843)(385.47573162,362.4746019)(385.02261353,362.65820312)
\curveto(384.57338877,362.84178903)(384.09487363,362.93358582)(383.58706665,362.93359375)
\curveto(382.58706263,362.93358582)(381.8214384,362.6269455)(381.29019165,362.01367188)
\curveto(380.75893946,361.40429047)(380.49331473,360.52147885)(380.49331665,359.36523438)
\curveto(380.49331473,358.21288741)(380.75893946,357.33007579)(381.29019165,356.71679688)
\curveto(381.8214384,356.10742077)(382.58706263,355.80273357)(383.58706665,355.80273438)
\curveto(384.09487363,355.80273357)(384.57338877,355.89453036)(385.02261353,356.078125)
\curveto(385.47573162,356.26171749)(385.90151245,356.53906096)(386.29995728,356.91015625)
\lineto(386.29995728,355.67382812)
\curveto(385.88588746,355.39257773)(385.44643478,355.18164044)(384.9815979,355.04101562)
\curveto(384.52065445,354.90039072)(384.03237369,354.83007829)(383.51675415,354.83007812)
\curveto(382.19253178,354.83007829)(381.14956407,355.23437477)(380.3878479,356.04296875)
\curveto(379.6261281,356.85546689)(379.2452691,357.96288766)(379.24526978,359.36523438)
\curveto(379.2452691,360.7714786)(379.6261281,361.87889937)(380.3878479,362.6875)
\curveto(381.14956407,363.4999915)(382.19253178,363.90624109)(383.51675415,363.90625)
\curveto(384.04018618,363.90624109)(384.53237319,363.83592866)(384.99331665,363.6953125)
\curveto(385.45815351,363.55858519)(385.89369995,363.35155415)(386.29995728,363.07421875)
}
}
{
\newrgbcolor{curcolor}{0 0 0}
\pscustom[linestyle=none,fillstyle=solid,fillcolor=curcolor]
{
\newpath
\moveto(468.73883057,481.2600708)
\lineto(468.73883057,480.36358643)
\lineto(467.70758057,480.36358643)
\curveto(467.32085878,480.36357821)(467.0513278,480.28545328)(466.89898682,480.12921143)
\curveto(466.75054685,479.9729536)(466.67632818,479.69170388)(466.67633057,479.28546143)
\lineto(466.67633057,478.7053833)
\lineto(468.45172119,478.7053833)
\lineto(468.45172119,477.86749268)
\lineto(466.67633057,477.86749268)
\lineto(466.67633057,472.1428833)
\lineto(465.59234619,472.1428833)
\lineto(465.59234619,477.86749268)
\lineto(464.56109619,477.86749268)
\lineto(464.56109619,478.7053833)
\lineto(465.59234619,478.7053833)
\lineto(465.59234619,479.16241455)
\curveto(465.59234488,479.89287555)(465.76226659,480.42412502)(466.10211182,480.75616455)
\curveto(466.44195341,481.0920931)(466.98101537,481.26006168)(467.71929932,481.2600708)
\lineto(468.73883057,481.2600708)
}
}
{
\newrgbcolor{curcolor}{0 0 0}
\pscustom[linestyle=none,fillstyle=solid,fillcolor=curcolor]
{
\newpath
\moveto(572.27038574,722.56109619)
\curveto(572.2703799,723.35406087)(572.10631756,723.975154)(571.77819824,724.42437744)
\curveto(571.45397447,724.87749685)(571.00670929,725.10405912)(570.43640137,725.10406494)
\curveto(569.86608543,725.10405912)(569.41686713,724.87749685)(569.08874512,724.42437744)
\curveto(568.76452403,723.975154)(568.60241482,723.35406087)(568.60241699,722.56109619)
\curveto(568.60241482,721.76812496)(568.76452403,721.14507871)(569.08874512,720.69195557)
\curveto(569.41686713,720.24273586)(569.86608543,720.01812671)(570.43640137,720.01812744)
\curveto(571.00670929,720.01812671)(571.45397447,720.24273586)(571.77819824,720.69195557)
\curveto(572.10631756,721.14507871)(572.2703799,721.76812496)(572.27038574,722.56109619)
\moveto(568.60241699,724.85211182)
\curveto(568.82897709,725.24273086)(569.11413306,725.53179307)(569.45788574,725.71929932)
\curveto(569.80553862,725.91069894)(570.2196007,726.00640197)(570.70007324,726.00640869)
\curveto(571.49694317,726.00640197)(572.1434269,725.68999604)(572.63952637,725.05718994)
\curveto(573.13951966,724.4243723)(573.38951941,723.59234188)(573.38952637,722.56109619)
\curveto(573.38951941,721.52984395)(573.13951966,720.69781353)(572.63952637,720.06500244)
\curveto(572.1434269,719.43218979)(571.49694317,719.11578386)(570.70007324,719.11578369)
\curveto(570.2196007,719.11578386)(569.80553862,719.20953377)(569.45788574,719.39703369)
\curveto(569.11413306,719.58843964)(568.82897709,719.87945497)(568.60241699,720.27008057)
\lineto(568.60241699,719.28570557)
\lineto(567.51843262,719.28570557)
\lineto(567.51843262,728.40289307)
\lineto(568.60241699,728.40289307)
\lineto(568.60241699,724.85211182)
}
}
{
\newrgbcolor{curcolor}{0 0 0}
\pscustom[linestyle=none,fillstyle=solid,fillcolor=curcolor]
{
\newpath
\moveto(351.71847534,602.96206665)
\curveto(351.41378429,602.18081804)(351.11690958,601.67105293)(350.82785034,601.43276978)
\curveto(350.53878516,601.1944909)(350.1520668,601.0753504)(349.66769409,601.0753479)
\lineto(348.80636597,601.0753479)
\lineto(348.80636597,601.97769165)
\lineto(349.43917847,601.97769165)
\curveto(349.73605159,601.97769324)(349.96652011,602.04800567)(350.13058472,602.18862915)
\curveto(350.29464478,602.32925539)(350.47628522,602.66128631)(350.67550659,603.1847229)
\lineto(350.86886597,603.6769104)
\lineto(348.21456909,610.13394165)
\lineto(349.35714722,610.13394165)
\lineto(351.40792847,605.00112915)
\lineto(353.45870972,610.13394165)
\lineto(354.60128784,610.13394165)
\lineto(351.71847534,602.96206665)
}
}
{
\newrgbcolor{curcolor}{0 0 0}
\pscustom[linestyle=none,fillstyle=solid,fillcolor=curcolor]
{
\newpath
\moveto(360.60128784,610.87808228)
\lineto(358.49777222,609.74136353)
\lineto(360.60128784,608.5987854)
\lineto(360.26144409,608.02456665)
\lineto(358.29269409,609.21401978)
\lineto(358.29269409,607.0050354)
\lineto(357.62472534,607.0050354)
\lineto(357.62472534,609.21401978)
\lineto(355.65597534,608.02456665)
\lineto(355.31613159,608.5987854)
\lineto(357.41964722,609.74136353)
\lineto(355.31613159,610.87808228)
\lineto(355.65597534,611.4581604)
\lineto(357.62472534,610.26870728)
\lineto(357.62472534,612.47769165)
\lineto(358.29269409,612.47769165)
\lineto(358.29269409,610.26870728)
\lineto(360.26144409,611.4581604)
\lineto(360.60128784,610.87808228)
}
}
{
\newrgbcolor{curcolor}{0 0 0}
\pscustom[linestyle=none,fillstyle=solid,fillcolor=curcolor]
{
\newpath
\moveto(318.86132812,929.390625)
\curveto(318.55663707,928.60937639)(318.25976237,928.09961128)(317.97070312,927.86132812)
\curveto(317.68163794,927.62304925)(317.29491958,927.50390875)(316.81054688,927.50390625)
\lineto(315.94921875,927.50390625)
\lineto(315.94921875,928.40625)
\lineto(316.58203125,928.40625)
\curveto(316.87890437,928.40625159)(317.10937289,928.47656402)(317.2734375,928.6171875)
\curveto(317.43749756,928.75781374)(317.61913801,929.08984466)(317.81835938,929.61328125)
\lineto(318.01171875,930.10546875)
\lineto(315.35742188,936.5625)
\lineto(316.5,936.5625)
\lineto(318.55078125,931.4296875)
\lineto(320.6015625,936.5625)
\lineto(321.74414062,936.5625)
\lineto(318.86132812,929.390625)
}
}
{
\newrgbcolor{curcolor}{0 0 0}
\pscustom[linestyle=none,fillstyle=solid,fillcolor=curcolor]
{
\newpath
\moveto(401.71847534,114.390625)
\curveto(401.41378429,113.60937639)(401.11690958,113.09961128)(400.82785034,112.86132812)
\curveto(400.53878516,112.62304925)(400.1520668,112.50390875)(399.66769409,112.50390625)
\lineto(398.80636597,112.50390625)
\lineto(398.80636597,113.40625)
\lineto(399.43917847,113.40625)
\curveto(399.73605159,113.40625159)(399.96652011,113.47656402)(400.13058472,113.6171875)
\curveto(400.29464478,113.75781374)(400.47628522,114.08984466)(400.67550659,114.61328125)
\lineto(400.86886597,115.10546875)
\lineto(398.21456909,121.5625)
\lineto(399.35714722,121.5625)
\lineto(401.40792847,116.4296875)
\lineto(403.45870972,121.5625)
\lineto(404.60128784,121.5625)
\lineto(401.71847534,114.390625)
}
}
{
\newrgbcolor{curcolor}{0 0 0}
\pscustom[linestyle=none,fillstyle=solid,fillcolor=curcolor]
{
\newpath
\moveto(407.10910034,123.74804688)
\lineto(407.10910034,120.49609375)
\lineto(406.11300659,120.49609375)
\lineto(406.11300659,123.74804688)
\lineto(407.10910034,123.74804688)
}
}
{
\newrgbcolor{curcolor}{0 0 0}
\pscustom[linestyle=none,fillstyle=solid,fillcolor=curcolor]
{
\newpath
\moveto(379.74917603,589.46234131)
\lineto(381.5128479,589.46234131)
\lineto(383.74526978,583.50921631)
\lineto(385.9894104,589.46234131)
\lineto(387.75308228,589.46234131)
\lineto(387.75308228,580.71429443)
\lineto(386.5987854,580.71429443)
\lineto(386.5987854,588.39593506)
\lineto(384.34292603,582.39593506)
\lineto(383.1534729,582.39593506)
\lineto(380.89761353,588.39593506)
\lineto(380.89761353,580.71429443)
\lineto(379.74917603,580.71429443)
\lineto(379.74917603,589.46234131)
}
}
\end{pspicture}

  \caption{Prinzipskizze einer Lochkamera }
  \label{fig:Lochkamera}
\end{figure}


Der Abstand von optischem Zentrum zur (virtuellen) Bildebene wird Fokallänge
$f$ genannt \ref{fig:Lochkamera}. Der Punkt $P$ wird durch die Projektion auf
den Sensor zum Punkt $P'$ mit den Koordinaten $(x^*,y^*)$. Da die Dreiecke
$C,M,P*$ und $C,0,P$ ähnlich sind gilt: \begin{equation}
  \frac{y^*}{f}=\frac{y}{b} \Rightarrow y^*=\frac{y}{b}*f \end{equation}
Genauso gilt für $x^*$: \begin{equation} x^*=\frac{x}{b}*f \end{equation}

Da der Ursprung des Sensorkoordinatensystems aufgrund von Konventionen und
Ungenauigkeiten bei der Kameramontage in den seltensten Fällen mit der
optischen Achse der Kamera übereinstimmt werden die projizierten Punkte noch um
$c_x$ und $c_y$ verschoben.

Eine weitere Besonderheit bei digitalen Kameras ist die Form der Pixel auf dem
Sensor. Da diese Rechteckig sind müssen im Modell verschiedene Werte $f_x$ und
$f_y$ für die Fokallänge angenommen werden.\cite{Bradski2008}

Wenn diese beiden Besonderheiten berücksichtigt und in
Vektorschreibweise dargestellt werden, ergibt sich \begin{equation}
  \begin{pmatrix} x^*\\y^* \end{pmatrix} = \begin{pmatrix} \frac{x}{b}*f_x+c_x
    \\ \frac{y}{b}*f_y+c_y \end{pmatrix} \end{equation}


Um die Projektion durch eine Matrixmultiplikation zu berechnen wird die
homogene Projektionsmatrix $Pr$ berechnet. Dazu werden die Vektoren
$P=\left(x,y,b\right)^T$ und $P^*=\left(x^*,y^*\right)^T$ zu den homogenen
Vektoren $\tilde{P}=\left(x,y,b,1\right)^T$ und
$\tilde{P^*}=\left(u^*,v^*,w^*\right)^T$ erweitert.

Die Berechnung der Projektion erfolgt jetzt durch \begin{equation}
  \tilde{P^*}=Pr*\tilde{P} \end{equation} \begin{equation} \tilde{P^*}=
  \begin{pmatrix} u^*\\v^*\\w^* \end{pmatrix} = \begin{bmatrix} f_x&0&c_x&0\\
    0&f_y&c_y&0\\ 0&0&1&0 \end{bmatrix} * \begin{pmatrix} x\\y\\b\\1
  \end{pmatrix} \end{equation}. \cite{Bradski2008}



Zur Umrechnung von homogenen in nicht homogene Koordinaten berücksichtigt man,
dass Punkte mit dem gleichen Verhältnis der Komponenten gleich sind. Außerdem
gilt für $w^*=1$

\begin{equation}
  \tilde{P^*}
  \begin{pmatrix}
    u^*\\v^*\\w^* \end{pmatrix} = \begin{pmatrix} x^*\\y^*\\1 \end{pmatrix}
\end{equation}.

Daraus folgt:

\begin{equation} P^*= \begin{pmatrix} x^*\\ y^* \end{pmatrix} = \frac{1}{w^*}*
  \begin{pmatrix} u^*\\ v^* \end{pmatrix} = \begin{pmatrix} \frac{u^*}{w^*}\\
    \frac{v^*}{w^*} \end{pmatrix} \end{equation}

Bei der Kalibrierung einer Kamera wird also die Projektionsmatrix $Pr$ oder die
Kameramatrix $K$ berechnet wobei gilt, dass \begin{equation} Pr=
  \begin{bmatrix} f_x&0&c_x&0\\ 0&f_y&c_y&0\\ 0&0&1&0 \end{bmatrix} = \left[K
  \left(0,0,0\right)^T\right] \end{equation}.

Der größte Nachteil bei Lochkameras ist die geringe Lichtmenge die durch das
Loch auf den Sensor fällt. Dadurch würde der Aufnahmevorgang für ein Bild sehr
lange brauchen bis genug Licht auf den Sensor gefallen ist.

Um diesen Nachteil auszugleichen werden in Kameras Linsen verwendet die das
Licht bündeln um mehr Licht auf den Senor fallen zu lassen. Durch Linsen werden
allerdings neue Abbildungscharakteristiken eingeführt, die im Modell
berücksichtigt werden müssen. Da es nur zwei Charakteristiken gibt, die einen
nennenswerten Einfluss auf das Bild haben, kann das Modell einfach erweitert
werden. Diese beiden Einflüsse entstehen aufgrund der Linsenform und der
Einbaulage des Sensors gegenüber der optischen Achse. 

Eine parabolische Linse oder ein komplexes Linsensystem hat nur einen geringen
Einfluss auf die Abbildung. Da aber häufig nur günstige sphärische Linsen
eingesetzt werden muss die radiale Verzeichnung korrigiert werden. Der Effekt
nimmt ausgehend vom optischen Zentrum zu und ist rotationssymetrisch dazu. Die
radiale Verzeichnung kann durch die ersten sieben Glieder einer Taylorreihe
$f(x)=a_0+a_1*x+a_2*x^2+\cdots+a_n*x^n$ beschrieben werden. Aus der Symmetrie
folgt, dass alle ungeraden Glieder gleich Null sind. Außerdem ist die
Verzeichnung am Mittelpunkt gleich Null. Deswegen ist auch $a_0=0$.

Die korrigierten Bildpunkte ergeben sich damit zu 
  \begin{align} x_{korr}=x*(1+k_1*r^2+k_2*r^4+k_3*r^6)\\
    y_{korr}=y*(1+k_1*r^2+k_2*r^4+k_3*r^6) \end{align}.

Die Verzerrung durch eine nicht parallele Einbaulage des Sensors zur Linse lässt
sich durch die zwei
Parameter $p_1$ und $p_2$ beschreiben \cite{Bradski2008}.

Der am \cob eingesetzte Kalibrieralgorithmus wurde von Zhang und Sturm
entwickelt und basiert auf der Erkennung eines definierten Musters. Dazu werden
Bilder von dem Objekt aufgenommen und wichtige Punkte - im Fall des \cob Ecken
eines Schachbretts - erkannt und zusammen mit einem Modell des Musters an den
Kalibrieralgorithmus übergeben. Dieser geht im ersten Schritt von einer
Lochkamera ohne Verzerrungen durch Linsen aus. Anschließend wird der
Parametervektor $\left[k_1,k_2,p_1,p_2,k_3\right]$ berechnet.\cite{zhang2000flexible}
Man erhält also alle wichtigen Parameter um einen Punkt im Raum auf einen Punkt auf dem Sensor
zu projizieren oder einem Punkt auf dem Sensor eine Linie im Raum zuzuordnen.

Außerdem kann dadurch die Pose eines bekannten Objekts im Raum berechnet
werden. 

% subsubsection Monokamerakalibrierung (end)

\subsubsection{Stereokamerakalibrierung} % (fold)
\label{sssec:Stereokamerakalibrierung} Ein Stereokamerasystem besteht, ähnlich
wie ein menschliches Augenpaar, aus zwei Kameras mit überlappendem Sichtfeld.
Mit den aus den zwei Kamerabildern gewonnenen Informationen über einen Punkt
lässt sich die genaue Position des Punktes im Raum berechnen. Bei einem idealen
Stereokamerasystem, wie in \ref{fig:DraufStereo} dargestellt, liegen beide
Bildebenen auf einer Ebene und die Transformation von einer zur anderen Kamera
kann durch eine Translation entlang der X-Achse ausgedrückt werden. Dadurch
sind auch alle Pixelreihen aneinander ausgerichtet. Wenn diese Bedingungen erfüllt sind, kann der Abstand
eines Punktes von der Ebene parallel zu den Bildebenen durch die optischen
Zentren der Kameras durch \begin{equation} Z=\frac{f*T}{a-b} \end{equation}
berechnet werden. Dabei ist $a$ die X-Koordinate des Punktes in Kamera 1 und 
$b$ die X-Koordinate des Punktes in Kamera 2.


\begin{figure}[htpb] \centering \def\svgwidth{\textwidth}
  \input{images/Stereo_Prinzip.pdf_tex} \caption{Draufsicht eines idealen
  Stereokamerapaares} \label{fig:DraufStereo} \end{figure}

Da in der Praxis keine ideale Ausrichtung der Kameras möglich ist, muss die
Transformation durch eine 6D Transformation angegeben werden. Aus dieser
Transformation und den intrinsischen Kameraparametern kann dann ein neues Bild
berechnet werden, in dem alle Bedingungen erfüllt sind.

Die Kalibrierung eines Stereokamerapaares benötigt also zusätzlich zu den
Parametern für eine Kamera die Transformation zwischen den Kameras bestehend
aus einem 3D Translationsvektor und einer 3x3 Rotationsmatrix.

Zur Kalibrierung werden wie bei der Monokamerakalibrierung Bilder von einem
definierten Objekt aufgenommen. Hierbei ist es wichtig, dass das Objekt auf
beiden Bildern zu sehen ist. Daraus wird die 6D Pose des Objekts für jede
Kamera berechnet. Wie in Figur~\ref{fig:StereoTrans} dargestellt ergibt sich
für die Rotationsmatrix

\begin{figure}[htpb] \centering \def\svgwidth{\textwidth}
  \input{images/Stereo_Transformation.pdf_tex} \caption{Transformationen im
  Stereokamerasystem} \label{fig:StereoTrans} \end{figure}


\begin{equation} R=R_1*(R_2)^T \end{equation} und für den Translationsvektor
\begin{equation} T=T_1-R*T_2 \end{equation}

Obwohl es möglich ist die gesuchte Transformation aus einem Bildpaar zu
berechnen, werden mehrere Paare aufgenommen um Bildrauschen und andere
Störeinflüsse zu kompensieren. Dazu wird der Median der gefundenen
Transformationen als Startwert für ein Optimierungsverfahren verwendet, das
zuverlässig die beste Lösung findet.

Mit den gefundenen Parametern können anschließend die Rectifizierungsmatrizen
für beide Kameras berechnet werden, die auf die Bilder angewendet ein Bildpaar
ergeben, bei den das alle Bedingungen für ein ideales Stereokamerasystem zutreffend
sind \cite{Bradski2008}.
% subsubsection Stereokamerakalibrierung (end)

% subsection Kamerakalibrierung (end) section Kalibrieralgorithmen (end)


\subsection{Kinematische Kalibrierung} % (fold) 

\label{sub:Kinematische Kalibrierung}

Zur Kinematischen Kalibrierung des Roboters wird ein Verfahren
verwendet, das auf dem Kalibrierverfahren für den PR2 von Vijay Pradeep basiert
\cite{Pradeep2010}.
Nach der Festlegung, welche Transformationen kalibriert werden sollen, werden
Pfade festgelegt, über die transformiert wird. Diese Pfade haben einen
gemeinsamen Ausgangspunkt und als Endpunkt entweder einen Sensor oder ein
detektierbares Objekt. Außerdem müssen alle vorher festgelegten
Transformationen in den Pfaden enthalten sein. In Abbildung \ref{fig:pfade}
sind zwei Pfade dargestellt. Pfad 1 geht vom Ursprungskoordinatensystem zu dem
detektierbaren Objekt und enthält neben der bekannten Koordinatentransformation
des Roboterarmes auch die zwei unbekannten Montagepositionen des Arms auf der
Basis und des Kalibrierobjekts am Arm. In Pfad 2 sind die Transformation von
der Basis zum Torso und vom Torso zur Kamera unbekannt. Die gesamten Pfade 1 und
2 sind in Grün dargestellt und setzen sich aus den roten festen Transformationen
und den schwarzen variablen Transformationen zusammen.

\begin{figure}[Htbp] \centering \def\svgwidth{0.7\textwidth}
  \input{images/Pfad_Kinematik.pdf_tex} \caption{Transformationspfade des \cob} \label{fig:pfade} \end{figure}

Im Anschluss werden Positionen für die in den Pfaden eingeschlossenen Aktoren
festgelegt, in denen das Kalibrierobjekt von allen Sensoren detektiert werden
kann. Außerdem müssen vorgegebene Transformationen und Konfigurationen für den
Optimierer bereitgestellt werden. Diese Schritte müssen für jeden Roboter
einmalig ausgeführt werden und sind direkt von der Hardware abhängig. 

Im darauf folgenden Datenaufnahmeschritt werden an jeder vorher definierten
Position die Sensordaten zusammen mit den aktuellen Gelenkwinkeln der Aktoren
aufgenommen. Diese Daten werden dann an den Optimierer übergeben.

Der Optimierer basiert auf dem Levenberg-Marquardt Verfahren\cite[Abschnitt 1]{pr2_estimation}.
Hierbei werden
für den \cob drei Durchläufe des Optimiers genutzt, wobei in jedem Schritt
weitere Parameter zur Optimierung freigegeben werden. Dadurch können lokale
Minima vermieden werden.

Zur Bestimmung des aktuellen Fehlers werden die während der Datenaufnahme
gespeicherten Gelenkwerte in ein Modell des Roboters eingesetzt. Dazu müssen
die \ac{DH-Parameter} der Aktoren sowie alle anderen Transformationen zumindest
näherungsweise bekannt sein. Durch das Modell können dann die markanten Punkte
des Kalibrierobjektes und die Position der Sensoren im Raum berechnet werden.
Die modellierten Punkte lassen sich anschließend über die Sensorparameter in
simulierte Messwerte umrechnen. Der Fehler der durch Modifikationen an einzelnen
Transformationen am Modell minimiert werden soll, ist die Differenz der
berechneten und der gemessenen Messwerte \cite{levi2012autonomous}.
Die Fehlerberechnung wird außerdem genutzt, um die Jacobimatrix für die Optimierung zu berechnen. Dazu werden alle
vorhandenen Parameter einzeln variiert um den Einfluss auf den Fehler zu berechnen.
Im nächsten Schritt werden die Parameter dann gezielt in die richtige Richtung
geändert \cite{forsyth2011}.

% subsection Kinematische Kalibrierung (end)




Laserscan - Erkennen von Calibration Pattern in 2d entfernungs slice Pattern
O-O -> Hough Circles, Hough Lines, Line mit geringstem Fehler zu den 2 Circles
ist Calibration Line --> Ausrichtung und abstand bekannt - Erkennen von
Calibration Pattern in Kamerabild --> 6DOF Pose bekannt - Optimierer zur
ausrichtung (6DOF)





%
% EOF
%pp
