\chapter{Problemstellung und Entwicklung einer Lösung}

Die Aufgabe, das vorhandene Kalibrierungsverfahren für alle \cob der dritten 
Generation anzupassen, konnte durch Tests an der Hardware und in Simulation 
auf zwei Bereiche, an denen Änderungen nötig waren, zurückgeführt werden. 
Mit den für die neuen Roboter erstellten Konfigurationsdateien konnte wegen
den fest einprogrammierten Werten keine gültige Kalibrierung berechnet werden.
Die vorhandenen Probleme waren in den Bereichen der Datenaufnahme
sowie der kinematischen Kalibrierung des Roboters angesiedelt. 

\section{Vorbereitung der Kalibrierung}
\label{sec:Vorbereitung der Kalibrierung}
Vor der Kalibrierung müssen die Gelenke des Torsos und des Arms in eine definierte
``Home'' Position gebracht werden. Dazu müssen diese Positionen manuell angefahren
und Anhand von Markierungen an den Gelenken überprüft werden. Danach muss aus 
den aktuellen Werten der Gelenke und den eingestellten Werten manuell ein neuer
Referenzwert berechnet und eingegeben werden. Fehler in diesem Schritt wirken 
sich auf die gesamte nachfolgende Kalibrierung aus und lassen sich nicht beheben.
Das Fehlerpotential das durch die manuelle Verarbeitung entsteht kann durch 
eine automatisierte Lösung gesenkt werden. Dadurch wird die Kalibrierung 
vereinfacht und beschleunigt.
\section{Datenaufnahme} % (fold)

\label{sec:Datenaufnahme}


Zur Datenaufnahme
wurden Armpositionen mit den dazugehörigen Torsostellungen in einer Datei 
abgelegt. Diese Datei wird anhand vorher festgeleter kartesischer Koordinaten
berechnet. Durch die unterschiedlichen Abmessungen der eingesetzten Arme 
ist es dem \ac{LWA} nicht möglich, alle dieser für den \ac{LBR3} festgelegten 
Positionen zu erreichen. Desweiteren ist eine Berechnung der Positionen
für andere Roboter, die sich wie der \raw\ durch den gesamten Aufbau vom 
\cob\ unterscheiden, nicht möglich, da dessen Kameras und Aktoren einen vollkommen
anderen Sicht- und Wirkungsbereich haben. Dadurch werden nicht genug Daten für die spätere 
Optimierung aufgenommen, was die Genauigkeit der Kalibrierung reduziert.
Außerdem mussten für die Kamerakalibrierung und die kinematische Kalibrierung 
des \cob\ zweimal nacheinander Daten von den selben Positionen des Arms und Torsos
aufgenommen werden.

\subsection{Bestimmung der Sample-Positionen}
\label{sub:Bestimmung der Sample-Positionen}

Eine Lösung für die unterschiedlichen Hardwarekonfigurationen ist die Erstellung
neuer Positionsdaten entweder durch manuelles Anfahren der Positionen und 
Abspeichern, oder durch das Festlegen eines Bereiches in dem das Kalibrierobjekt
bewegt und von den Kameras erkannt werden kann. In diesem Raum können dann
beliebig viele gültige Positionen berechnet werden. Eine weitere Möglichkeit
ist das Festlegen eines spiralförmigen Pfades der vom Roboterarm abgefahren 
und regelmäßig auf die Sichtbarkeit des Kalibrierobjekts überprüft wird. Hierdurch
wird ein sichtbares Feld abgesteckt und gleichzeitig Samples für die 
Optimierung aufgenommen. Bei der dritten Möglichkeit werden im voraus Positionen
Anhand einer Strategie berechnet und abgespeichert.

\subsubsection{Manuelles Einlernen}
\label{ssub:Manuelles Einlernen}
Die Lösung, Positionen durch manuelles Einlernen zu bestimmen, wurde verworfen,
da dieser sehr zeitaufwendige Schritt für jeden neuen Roboter nötig gewesen wäre.

Der Unterschied der beiden anderen Möglichkeiten besteht im dem Zeitpunkt der 
Berechnung der inversen Kinematik des Armes und des Torsos.

\subsubsection{Abfahren und Überprüfen eines Pfades}
\label{ssub:Abfahren und Überprüfen eines Pfades}

Bei der Lösung des 
definierten Bereichs in dem Samples aufgenommen werden kann die Qualität und die
Position des Arms ohne die eigentliche Hardware in Simulation berechnet werden, 
während bei dem Abfahren eines Pfades die Gelenkwinkel unmittelbar vor der 
Bewegung berechnet werden. Der Vorteil dieser Methode ist, dass zur Berechnung
der Samplepositonen nur ein grobes Modell der Lage der optischen Achsen benötigt
wird, das aus den ungenauen CAD-Daten berechnet werden kann. Aus diesem Modell
kann dann der Ursprung des Pfades bestimmt werden. 

Problematisch ist, dass Kollisionen mit dem Roboter oder der Umwelt erst bei
der Bewegung des Arms erkannt und durch einen Abbruch der Datenaufnahme 
verhindert werden. Eine kollisionsfreie geplante Bewegung kann wegen der zu 
ungenauen Kalibrierung noch nicht bestimmt werden.

\subsubsection{Berechnen und Überprüfen der Samples in Simulation}
\label{ssub:Berechnen und Überprüfen der Samples in Simulation}

Der Vorteil bei der Berechnung und Speicherung der
Gelenkwinkel in der Simulation ist, dass weniger ungültige Samples angefahren werden
und damit der Zeitaufwand der Datenaufnahme reduziert wird. Außerdem können alle
Trajektorien des Arms und des Torsos in der Simulation auf Kollisionen überprüft
werden. Eine rechnerische Überprüfung auf Kollisionen ist wegen der noch nicht 
erfolgten Kalibrierung noch nicht zuverlässig möglich. Ein Nachteil ist, dass
Modelle zur Strategie der Berechnung erzeugt werden müssen die genug Toleranz für Fehler zulassen um die 
noch nicht erfolgte Kalibrierung auszugleichen.

Die Entscheidung fiel wegen der erheblich sichereren Durchführung für die Berechnung
und manuelle Überprüfung der Samples in Simulation. 



\subsection{Doppelte Datenaufnahme vermeiden}
\label{sub:Doppelte Datenaufnahme vermeiden}

Die doppelte Datenaufnahme war notwendig, da der Optimierer für die kinematische
Kalibrierung bereits verarbeitete, also entzerrte Bilder des
Kalibrierungsobjekts und die Projektionsmatrix der Kameras benötigt, um den Fehler
zu bestimmen. Dazu wurden nur die Daten des zweiten Aufnahmeschritts für die 
kinematische Kalibrierung des Roboters verwendet. Für die Aufnahme wurden unter
anderem die Nachrichten die die Kameraparameter und die Kamerabilder enthalten
aufgezeichnet und in einer Datei gespeichert. Die Entzerrung der Bilddaten kann
aber auch nach den angegebenen Formeln \footnote{Siehe Kapitel \ref{sub:Kamerakalibrierung}} erst bei
der Optimierung erfolgen. Genauso können die Kameraparameter auch nachträglich
aus der entsprechenden Kalibrierungsdatei gelesen werden. Durch diese Modifikationen
können alle für die Kalibrierung notwendigen Daten in einem Aufnahmeschritt 
erfasst werden.

Durch den fehlenden Aufnahmeschritt kann der Zeitaufwand der Datenaufnahme 
und damit der gesamten Kalibrierung reduziert werden.
Weil Armbewegungen ohne Kollisionsberechnung zu Schäden an der Hardware und der
Umwelt führen können kann außerdem die Sicherheit der Kalibrierung erhöht werden.

% section Datenaufnahme (end)


\section{Optimierung} % (fold)

\label{sec:Optimierung}

Für die Optimierung wird aus den Konfigurationsdateien ein Modell des zu 
kalibrierenden Roboters erstellt. Dafür müssen die \ac{DH-Parameter} der Aktoren
sowie alle anderen Transformationen bekannt sein. Außerdem muss zu jeder Transformation 
angegeben werden, ob sie kalibrierbar oder festgelegt ist.

\subsection{Doppelte Modellierung des Roboters} % (fold)
\label{sub:doppelte modellierung}


Da das \ac{ROS}-Modell eines 
Roboters nicht nur zur Berechnung der Transformationen verwendet wird, sondern für die Visualisierung auch genaue
Informationen über die Lage eines Gelenks benötigt, sind keine \ac{DH-Parameter} sondern
genaue Transformationen mit Rotations- und Translationsvektor angegeben. Die für die
Kalibrierung benötigten \ac{DH-Parameter} wurden bisher manuell berechnet und
in die \texttt{system.yaml} eingegeben. Die Beschränkungen der \ac{DH-Parameter}
sind dadurch auch am Roboter vorhanden. Außerdem gibt es nicht für alle am \cob\ 
eingesetzten Komponenten ausreichend genaue \ac{DH-Parameter}. Die bisher manuell
ausgeführte Berechnung der \ac{DH-Parameter} sollte deswegen automatisiert oder
ersetzt werden. Dazu bieten sich drei Möglichkeiten, die evaluiert wurden. 




\subsubsection{Berechnung der \ac{DH-Parameter}} % (fold)

\label{ssub:Berechnung der dh-par}

Um die Kalibrierung auch für andere Roboter zu nutzen,
besteht die Möglichkeit die Parameter automatisiert berechnen zu lassen. Der Vorteil
dieser Methode liegt in der einfachen implementierung. Durch die Berechnung der Parameter 
wird eine neue Konfigurationsdatei erzeugt mit der der Roboter kalibriert werden kann.

Die Nachteile dieser Methode liegen in der Berechnung der Vorwärtskinematik während der
Optimierung sowie der weiteren Verwendung von \ac{DH-Parameter}n. Dadurch 
werden alle Nachteile die sich durch die vereinfachte Beschreibung einer 
Transformation durch \ac{DH-Parameter} ergeben auf den Roboter übertragen.
Diese Vereinfachungen lassen nur reine Rotations- und Translationsgelenke zu.
Da ein möglichst universell einsetzbares Verfahren implementiert werden soll, 
kann nicht davon ausgegangen werden, dass alle zu kalibrierenden Roboter diese 
Anforderungen erfüllen. 

Die Berechnung der Vorwärtskinematik während der Optimierung wirkt sich auf die 
Laufzeit der Kalibrierung aus. Die Umrechnung der aufgenommenen Gelenkwinkel mit 
den gegebenen \ac{DH-Parameter}n erfolgt etwa 250.000 mal. Durch den häufigen 
Funktionsaufruf wirken sich Änderungen in der Rechenzeit stark auf die 
Gesamtrechenzeit aus.

% subsubsection Berechnung der \ac{DH-Parameter} (end)
\subsubsection{Berechnen der Vorwärtskinematik mit dem ROS-Service} % (fold)

\label{ssub:Berechnen der Vorwärtskinematik mit dem ROS-Service}

Eine Alternative hierzu ist die Nutzung des ROS-Services zur Berechnung der Vorwärtskinematik.
Hierzu werden weiterhin die Gelenkwinkel erfasst und abgelegt. Während der Optimierung werden
die Gelenkwinkel an den ROS-Node der für die Berechnungen der Vorwärtskinematik zuständig ist 
übergeben und die Transformation berechnet. Dadurch müssen in den Konfigurationsdateien keine
\ac{DH-Parameter} abgelegt sein und es lassen sich alle modellierten Arten von Aktoren berechnen.

Problematisch hierbei ist, dass es für jeden Aktor einen Service geben muss, der die 
Vorwärtskinematik berechnen kann. Für den \cob gibt es diesen Service nur für den Arm.
Für Tests wurde dieser Node angepasst um auch Lösungen für den Torso zu berechnen.
Dabei kann festgestellt werden, dass der Zeitbedarf der Kalibrierung ansteigt.
Die berechnung der Transformationen erfolgt hierbei wie mit \ac{DH-Parameter}n
während der Optimierung. Durch die Nutzung der \ac{ROS}-Services zur Berechnung 
der Transformationen und den damit verbundenen zusätzlichen Funktionsaufrufen 
erhöht sich der Overhead und damit auch die Gesamtrechenzeit. 
% subsubsection Berechnen der Vorwärtskinematik mit dem ROS-Service (end)


\subsubsection{Abspeichern der Transformationen während der Datenaufnahme} % (fold)
\label{ssub:Abspeichern der Transformationen während der Datenaufnahme}

% subsubsection Abspeichern der Transformationen während der Datenaufnahme (end)

 Die dritte Möglichkeit erfordert Änderungen sowohl an der Datenaufnahme als auch an der Optimierung.
 Anstatt Gelenkwinkel aufzunehmen werden hierbei die Transformationen eines Aktors während der Datenaufnahme
 berechnet. An dieser Stelle muss die Berechnung nur einmal pro Sample, also insgesamt etwa 30 bis 50 mal 
 erfolgen. Die in \ref{ssub:Berechnung der dh-par} angegebenen 250.000 Berechnungen werden für die Berechnung 
 der Kovarianz- und der Jacobi-Matrix benötigt. Bei dieser Möglichkeit werden hierzu nicht die Gelenkwinkel,
 sondern die Rotations- und Translationsvektorelemente variiert.
 Der Vorteil dieser Methode ist neben der verkürzten Rechenzeit, dass für alle mit \ac{ROS} verwendeten 
 Roboter ein Modell vorhanden sein muss das hier verwendet werden kann. Die Modellierung des Manipulators
 muss also nicht zusätzlich in \ac{DH-Parameter}n vorhanden sein. Außerdem 
 bietet \ac{ROS} ohne zusätzliche Nodes die Möglichkeit die aktuellen Transformationen
 zu berechnen. Die Transformation kann also für jeden Roboter ohne zusätzliche
 Änderungen berechnet werden.
 
 Dafür muss schon bei der Datenaufnahme bekannt sein welche Transformationen für die Optimierung benötigt 
 werden. Das bedeutet, dass die in Abbildung \ref{fig:pfade} dargestellten Pfade bereits genau
 definiert sein müssen und nur wenige Änderungen nachträglich gemacht werden können. Bei der 
 Aufnahme von Gelenkwinkeln ist diese Festlegung erst bei der Kalibrierung von Bedeutung. 
 Wenn nachträglich Änderungen an den angegebenen Pfaden gemacht werden müssen 
 erneut Daten aufgenommen werden. Bei den anderen Methoden ist eine Änderung nach
 der Datenaufnahme möglich. Dafür wird durch die gezielte Aufnahme von Daten der
 Speicherbedarf verringert.

 Da sich bei dieser Methode die Dateistruktur ändert müssen viele Teile zur 
 Berechnung des Modells des Optimierers angepasst oder neu entwickelt werden. 
% subsection Denavit-Hartenberg Parameter (end)


Umgesetzt wurde trotz erheblichem Mehraufwand die letzte Alternative, die dafür
die am universellsten einsetzbarste Alternative ist. 

\subsection{Verkettung von Aktoren}
\label{sub:Verkettung von Aktoren}

Durch das Format der Parameterdateien des bisherigen Kalibrieralgorithmus konnten
nicht alle Eigenschaften des \cob\ abgebildet werden. Dazu gehören Aktoren, die
an anderen Aktoren montiert sind. Beim \cob\ führt das zu Einschränkungen die 
die Kopfachse betreffen. Beim \cob\ ist der Kopf mit einer Achse an einer 
unbekannten Positon auf dem Torso befestigt. Für die bisherige Kalibrierung wurde
um dies zu umgehen die Kopfachse als statische Transformation angesehen. Dadurch
wurde eine Kalibrierung in einem bestimmten Bereich ermöglicht. Zur Steigerung
der Genauigkeit sollte die Kalibrierung den gesamten Arbeitsraum des Roboters
abdecken. 

Um dies zu ermöglichen muss der Dateiaufbau der Konfigurationsdatei 
\texttt{sensors.yaml} und das Erzeugen des Modells für die Optimierung geändert
werden.
% section Optimierung (end)
