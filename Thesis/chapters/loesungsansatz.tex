\chapter{Problemstellung und Entwicklung einer Lösung}

Die Aufgabe, das vorhandene Kalibrierungsverfahren an alle \cob\ der dritten 
Generation anzupassen und zu generalisieren erfordert eine Analyse des
bestehenden Verfahrens. Mit dem Ansatz, für einen neuen Roboter neue Konfigurationsdateien 
mit fest einprogrammierten Werten konnte keine genaue Kalibrierung berechnet werden.
Die vorhandenen Probleme waren in den Bereichen der Datenaufnahme
sowie der kinematischen Kalibrierung des Roboters angesiedelt. 
Außerdem sollten Lösungen gefunden werden, die Montagepositionen der 
Laserscanner und das Tablett zu Kalibrieren.


\section{Vorbereitung der Kalibrierung}
\label{sec:Vorbereitung der Kalibrierung}
Vor der Kalibrierung müssen die Gelenke des Torsos und des Arms in eine definierte
"`Home"'-Position gebracht werden. Dazu müssen diese Positionen manuell angefahren
und anhand von Markierungen an den Gelenken überprüft werden. Danach muss aus 
den aktuellen Werten der Gelenke und den eingestellten Werten manuell ein neuer
Referenzwert berechnet und eingegeben werden. Fehler in diesem Schritt wirken 
sich auf die gesamte nachfolgende Kalibrierung aus und lassen sich nicht beheben.
Das Fehlerpotential, das durch die manuelle Verarbeitung entsteht, kann durch 
eine automatisierte Lösung gesenkt werden. Dadurch wird die Kalibrierung 
vereinfacht und beschleunigt.


\section{Datenaufnahme} % (fold)

\label{sec:Datenaufnahme}

Zur Datenaufnahme
werden Armpositionen mit den dazugehörigen Torsostellungen in einer Datei 
abgelegt. Diese Datei wird anhand vorher festgeleter kartesischer Koordinaten
berechnet. Durch die unterschiedlichen Abmessungen der eingesetzten Arme 
ist es dem \ac{LWA} nicht möglich, alle dieser für den \ac{LBR3} festgelegten 
Positionen zu erreichen. Desweiteren ist eine Berechnung der Positionen
für andere Roboter, die sich wie der \raw\ durch den gesamten Aufbau vom 
\cob\ unterscheiden, nicht möglich, da dessen Kameras und Aktoren einen vollkommen
anderen Sicht- und Wirkungsbereich haben. Es werden nicht genügend Daten für die spätere 
Optimierung aufgenommen, was die Genauigkeit der Kalibrierung reduziert.
Außerdem mussten für die Kamerakalibrierung und die kinematische Kalibrierung 
des \cob\ zweimal nacheinander Daten von denselben Positionen des Arms und des Torsos
aufgenommen werden.

\subsection{Bestimmung der Sample-Positionen}
\label{sub:Bestimmung der Sample-Positionen}

Eine Lösung für die unterschiedlichen Hardwarekonfigurationen ist die Erstellung
neuer Positionsdaten entweder durch manuelles Einlernen, automatisches Einlernen
oder berechnen und überprüfen in Simulation.

%durch das 
%Festlegen eines Bereiches in dem das Kalibrierobjekt
%bewegt und von den Kameras erkannt werden kann. In diesem Raum können dann
%beliebig viele gültige Positionen berechnet werden.

%Die zweite Möglichkeit
%ist, das Festlegen eines spiralförmigen Pfades, der vom Roboterarm 

%Nachfolgend werden diese drei Möglichkeiten beschrieben.

\begin{description}
\item[Manuelles Einlernen]
    \label{ssub:Manuelles Einlernen}
    Zum manuellen Einlernen der Samplepositionen werden die Aktoren des Roboters
    von einem Bediener in Positionen gefahren, an denen das am Arm montierte
    Kalibrierobjekt in den Kameras sichtbar ist. Anschließend werden die zu den
    Positionen gehörenden Gelenkwinkel für spätere Kalibrierungen gespeichert.
    Das manuelle Einlernen der Positionen ist für jeden Roboter durchführbar, 
    aber sehr aufwendig.


%Der Unterschied der beiden anderen Möglichkeiten besteht im dem Zeitpunkt der 
%Berechnung der inversen Kinematik des Armes und des Torsos.

  \item[Automatisches Einlernen]
    \label{ssub:Abfahren und Überprüfen eines Pfades}
    Hierbei wird ausgehend von der optischen Achse einer Kamera ein spiralförmiger
    Pfad fesgelegt, der vom Roboterarm abgefahren 
    und regelmäßig auf die Sichtbarkeit des Kalibrierobjekts überprüft wird. Hierdurch
    wird ein sichtbares Feld abgesteckt und gleichzeitig Samples für die 
    Optimierung aufgenommen.

    Ein Vorteil dieser Lösung ist, dass zur Berechnung
    der Samplepositonen nur ein grobes Modell der Lage der optischen Achsen benötigt
    wird, das aus den ungenauen CAD-Daten berechnet werden kann. Aus diesem Modell
    kann dann der Ursprung des Pfades bestimmt werden. 

    Problematisch ist, dass Kollisionen mit dem Roboter oder der Umwelt erst bei
    der Bewegung des Arms erkannt und durch einen Abbruch der Datenaufnahme 
    verhindert werden. Eine kollisionsfreie geplante Bewegung kann wegen der zu 
    ungenauen Kalibrierung noch nicht bestimmt werden.




  \item[Berechnen und Überprüfen in Simulation]
    \label{ssub:Berechnen und Überprüfen der Samples in Simulation}

    Es werden im Voraus Positionen anhand einer Strategie berechnet und abgespeichert.

    Der Vorteil bei der Berechnung und Speicherung der
    Gelenkwinkel in der Simulation ist, dass weniger ungültige Samples angefahren werden
    und damit der Zeitaufwand der Datenaufnahme reduziert wird. Außerdem können alle
    Trajektorien des Arms und des Torsos in der Simulation auf Kollisionen überprüft
    werden.
    Eine rechnerische Überprüfung auf Kollisionen ist wegen der noch nicht 
    erfolgten Kalibrierung nicht zuverlässig möglich. 
    
    Ein Nachteil ist, dass
    Modelle zur Strategie der Berechnung erzeugt werden müssen, die genug Toleranz für Fehler zulassen, um die 
    nicht erfolgte Kalibrierung auszugleichen.

\end{description}
Die Lösung, Positionen durch manuelles Einlernen zu bestimmen, wurde verworfen,
da diese für jeden neuen Roboter zeitaufwendige Schritte bedeutet.


%Bei der Lösung des 
%definierten Bereichs, in dem Samples aufgenommen werden, kann die Qualität der 
%Samplepositionen und die Position des Arms ohne die eigentliche Hardware durch
%die Simulation berechnet werden, 
Der Unterschied der beiden anderen ist der Zeitpunkt, wann die Berechnung 
der Gelenkwinkel ausgeführt wird. Während bei dem Abfahren eines Pfades die Gelenkwinkel unmittelbar vor der 
Bewegung berechnet werden, werden sie bei der Simulation schon vorher berechnet.
%können die berechneten Samples in der Simulation 
%überprüft werden.

Die Entscheidung fiel wegen der erheblich sichereren Durchführung für die Berechnung
der Samples, die durch manuelle Überprüfung bestätigt werden können. 



\subsection{Doppelte Datenaufnahme vermeiden}
\label{sub:Doppelte Datenaufnahme vermeiden}

Die doppelte Datenaufnahme war notwendig, da der Optimierer für die kinematische
Kalibrierung bereits verarbeitete, also entzerrte Bilder des
Kalibrierungsobjekts und die Projektionsmatrix der Kameras benötigt, um den Fehler
zu bestimmen. Dazu wurden nur die Daten des zweiten Aufnahmeschritts für die 
kinematische Kalibrierung des Roboters verwendet. Für die Aufnahme wurden unter
anderem die Nachrichten, die die Kameraparameter und die Kamerabilder enthalten,
aufgezeichnet und in einer Datei gespeichert. Die Entzerrung der Bilddaten kann
aber auch nach den angegebenen Formeln\footnote{Siehe Kapitel \ref{sub:Kamerakalibrierung}} erst bei
der Optimierung erfolgen. Genauso können die Kameraparameter auch nachträglich
aus der entsprechenden Kalibrierungsdatei gelesen werden. Durch diese Modifikationen
können alle für die Kalibrierung notwendigen Daten in einem Aufnahmeschritt 
erfasst werden.

Durch den fehlenden Aufnahmeschritt kann der Zeitaufwand der Datenaufnahme 
und damit der gesamten Kalibrierung reduziert werden.
Weil Armbewegungen ohne Kollisionsberechnung zu Schäden an der Hardware und der
Umwelt führen können, kann durch die geringere Anzahl an Bewegungen außerdem 
die Sicherheit der Kalibrierung erhöht werden.

% section Datenaufnahme (end)


\section{Optimierung} % (fold)

\label{sec:Optimierung}

Für die Optimierung wird aus den Konfigurationsdateien ein Modell des zu 
kalibrierenden Roboters erstellt. Dafür müssen die \ac{DH-Parameter} der Aktoren
sowie alle anderen Transformationen bekannt sein. Außerdem muss zu jeder Transformation 
angegeben werden, ob sie kalibrierbar oder festgelegt ist.

\subsection{Doppelte Modellierung des Roboters} % (fold)
\label{sub:doppelte modellierung}


Da das \ac{ROS}-Modell eines Roboters nicht nur zur Berechnung der 
Transformationen verwendet wird, sondern für die Visualisierung auch genaue
Informationen über die Lage eines Gelenks benötigt, sind keine \ac{DH-Parameter} sondern
genaue Transformationen mit Rotations- und Translationsvektor angegeben. Die für die
Kalibrierung benötigten \ac{DH-Parameter} wurden bisher manuell berechnet und
in die \texttt{system.yaml} eingegeben. Die Beschränkungen der \ac{DH-Parameter}
sind dadurch auch am Roboter vorhanden. Außerdem gibt es nicht für alle am \cob\ 
eingesetzten Komponenten ausreichend genaue \ac{DH-Parameter}. Die bisher manuell
ausgeführte Berechnung der \ac{DH-Parameter} soll deswegen automatisiert oder
ersetzt werden.

Dazu bieten sich die automatische 
Berechnung der \ac{DH-Parameter}, die Berechnung der Vorwärtskinematik 
durch den entsprechenden \ac{ROS}-Service oder das Abspeichern 
der Transformationen während der Datenaufnahme an:

\begin{description}
  \item[Berechnung der \ac{DH-Parameter}] % (fold)

    \label{ssub:Berechnung der dh-par}

    Um die Kalibrierung auch für andere Roboter zu nutzen,
    besteht die Möglichkeit, die Parameter automatisiert berechnen zu lassen.
    
    Der Vorteil
    dieser Methode liegt in der einfachen Implementierung. Durch die Berechnung der Parameter 
    wird eine neue Konfigurationsdatei erzeugt, mit der der Roboter kalibriert werden kann.

    Die Nachteile dieser Methode liegen in der Berechnung der Vorwärtskinematik während der
    Optimierung sowie in der weiteren Verwendung der \ac{DH-Parameter}. Dadurch 
    werden alle Nachteile, die sich durch die vereinfachte Beschreibung einer 
    Transformation durch \ac{DH-Parameter} ergeben, auf den Roboter übertragen.
    Diese Vereinfachungen lassen nur reine Rotations- und Translationsgelenke zu.

    Die Berechnung der Vorwärtskinematik während der Optimierung wirkt sich auf die 
    Laufzeit der Kalibrierung aus. Die Umrechnung der aufgenommenen Gelenkwinkel mit 
    den gegebenen \ac{DH-Parameter}n erfolgt etwa 250.000 mal. Durch den häufigen 
    Funktionsaufruf wirken sich Änderungen in der Rechenzeit stark auf die 
    Gesamtrechenzeit aus.
% subsubsection Berechnung der \ac{DH-Parameter} (end)
  \item[Berechnen der Vorwärtskinematik mit dem ROS-Service] % (fold)

    \label{ssub:Berechnen der Vorwärtskinematik mit dem ROS-Service}

    Eine Alternative hierzu ist die Nutzung des ROS-Services zur Berechnung der
    Vorwärtskinematik während der Optimierung.
    Dazu werden weiterhin die Gelenkwinkel erfasst und abgelegt. Es werden
    die Gelenkwinkel an den ROS-Node, der für die Berechnungen der
    Vorwärtskinematik zuständig ist, 
    übergeben und die Transformation berechnet.
    
    Dadurch müssen in den
    Konfigurationsdateien keine
    \ac{DH-Parameter} abgelegt sein und es lassen sich alle modellierten Arten
    von Aktoren berechnen.

    Problematisch hierbei ist, dass es für jeden Aktor einen Service geben
    muss, der die Vorwärtskinematik berechnen kann. Für den \cob\ gibt es diesen Service nur für den Arm.
    Für Tests wurde dieser Node angepasst, um auch Lösungen für den Torso zu berechnen.
    Dabei konnte festgestellt werden, dass der Zeitbedarf der Kalibrierung ansteigt.
    Die Berechnung der Transformationen erfolgt hierbei wie mit \ac{DH-Parameter}n
    während der Optimierung.
    Durch die Nutzung der \ac{ROS}-Services zur Berechnung 
    der Transformationen und den damit verbundenen zusätzlichen Funktionsaufrufen 
    erhöht sich der Overhead und damit auch die Gesamtrechenzeit. 
% subsubsection Berechnen der Vorwärtskinematik mit dem ROS-Service (end)


  \item[Abspeichern der Transformationen während der Datenaufnahme] % (fold)
    \label{ssub:Abspeichern der Transformationen während der Datenaufnahme}

% subsubsection Abspeichern der Transformationen während der Datenaufnahme (end)

    Diese Möglichkeit erfordert Änderungen sowohl an der Datenaufnahme als auch an der Optimierung.
    Anstatt Gelenkwinkel aufzunehmen, werden hierbei die Transformationen eines Aktors während der Datenaufnahme
    berechnet. An dieser Stelle muss die Berechnung nur einmal pro Sample, also insgesamt etwa 30 bis 50 mal 
    erfolgen. Es werden 250.000 Durchläuf für die Berechnung der Kovarianz- und
    der Jacobi-Matrix benötigt. Bei dieser Möglichkeit werden hierzu nicht die Gelenkwinkel,
    sondern die Rotations- und Translationsvektorelemente variiert.

    Der Vorteil dieser Methode ist neben der verkürzten Rechenzeit, dass für alle mit \ac{ROS} verwendeten 
    Roboter ein Modell vorhanden sein muss, das hier verwendet werden kann. Die Modellierung des Manipulators
    muss also nicht zusätzlich in \ac{DH-Parameter}n vorhanden sein. Außerdem 
    bietet \ac{ROS} ohne zusätzliche Nodes die Möglichkeit, die aktuellen Transformationen
    zu berechnen. Die Transformation kann also für jeden Roboter ohne zusätzliche
    Änderungen berechnet werden.

    Dafür muss bereits bei der Datenaufnahme bekannt sein, welche Transformationen für die Optimierung benötigt 
    werden. Das bedeutet, dass die in Abbildung \ref{fig:pfade} dargestellten Pfade bereits genau
    definiert sein müssen und nur wenige Änderungen nachträglich gemacht werden können. Bei der 
    Aufnahme von Gelenkwinkeln ist diese Festlegung erst bei der Kalibrierung von Bedeutung. 
    Wenn nachträglich Änderungen an den angegebenen Pfaden gemacht werden, müssen 
    erneut Daten aufgenommen werden. Bei den anderen Methoden ist eine Änderung nach
    der Datenaufnahme möglich.

    Bei dieser Methode ändert sich die Dateistruktur, müssen viele Teile zur 
    Berechnung des Modells des Optimierers angepasst oder neu entwickelt werden. 
% subsection Denavit-Hartenberg Parameter (end)
\end{description}


Da ein möglichst universell einsetzbares Verfahren implementiert werden soll, 
kann nicht davon ausgegangen werden, dass alle zu kalibrierenden Roboter die 
durch die weitere Verwendung von \ac{DH-Parameter}n entstehenden Anforderungen
erfüllen. 

Trotz dem erheblichen Mehraufwand wird die dritte Alternative umgesetzt. Verglichen 
mit den anderen Lösungen ist diese, durch die Aufhebung der Beschränkungen der 
\ac{DH-Parameter} und den Verzicht auf zusätzliche Services, die universellste.
Zudem müssen für andere neue Funktionen 
Änderungen am Modell des Optimierers vorgenommen werden, die die gleichen 
Funktionen betreffen.

\subsection{Verkettung von Aktoren}
\label{sub:Verkettung von Aktoren}

Durch das Format der Parameterdateien des bisherigen Kalibrieralgorithmus konnten
nicht alle Eigenschaften des \cob\ abgebildet werden. Dazu gehören Aktoren, die
an anderen Aktoren montiert sind. Beim \cob\ führt das zu Einschränkungen, die 
die Kopfachse betreffen. Beim \cob\ ist der Kopf mit einer Achse an einer 
unbekannten Positon auf dem Torso montiert. Für die bisherige Kalibrierung wurde,
um dies zu umgehen, die Kopfachse als statische Transformation angesehen. Dadurch
wurde eine Kalibrierung in einem bestimmten Bereich ermöglicht. Zur Steigerung
der Genauigkeit sollte die Kalibrierung den gesamten Arbeitsraum des Roboters
abdecken. 

Um dies zu ermöglichen, muss der Dateiaufbau der Konfigurationsdatei\break 
\texttt{sensors.yaml} und das Erzeugen des Modells für die Optimierung geändert
werden.
% section Optimierung (end)


\section{Extrinsische Kalibrierung der Laserscanner}
\label{sec:extr Kalibrierung der Laserscanner}


Für die Kalibrierung von Laserscannern gibt es je nach Ausführung der Scanner
verschiedene Verfahren zur Kalibrierung. 3D Laserscanner, die zusätzlich zu den
Entfernungsdaten auch Intensitätswerte der reflektierten Strahlen liefern,
können wie Kameras behandelt werden. Die Intensitäten lassen sich zu einem Bild
zusammensetzen auf dem, wie auf dem Kamerabild, das Kalibrierobjekt erkannt
werden kann. Die 6\ac{DOF} Pose des Objekts lässt sich durch die gemessenen 
Entfernungen noch genauer berechnen.\cite{Pradeep2010}

Da die am \cob\ eingesetzten Laserscanner nur 2D Scanner sind, die keine 
Helligkeitswerte messen, muss ein anderes Verfahren gefunden werden, mit dem
die Position der Laserscanner kalibriert werden kann.
Die Kalibrierung der Montageposition besteht auch hier aus dem Erkennen eines
Kalibrierobjektes und der Optimierung der Parameter. 
