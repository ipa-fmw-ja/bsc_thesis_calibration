%
% Diplomarbeit mit LaTeX
% ===========================================================================
% This is part of the book "Diplomarbeit mit LaTeX".
% Copyright (c) 2002-2005,2007 Tobias Erbsland
% Copyright (c) 2005 Andreas Nitsch
% See the file diplomarbeit_mit_latex.tex for copying conditions.
%

%\begin{titlepage}
	%\vspace*{6cm}
	%\begin{center}
		%\Huge
		%Automatische Kalibrierung eines mobilen Serviceroboters\\
		%\vspace{1cm}
		%\large
		%\today\\
		%\vspace{2cm}
		%Jannik Abbenseth
		
	%\end{center}
	%\normalsize
	
%\end{titlepage}

\begin{titlepage}
  \newgeometry{left=30mm, right=30mm}
  \title{Bachelor-Thesis}
  
\subtitle{Automatische Kalibrierung eines mobilen Serviceroboters}
\author{Jannik Sven Abbenseth}
%\publishers{Hochschule Furtwangen University, Fakultät für Maschinenbau und Mechatronik
  %\\Fraunhofer Gesellschaft, Institut für Produktionstechnik und Automatisierung}
\publishers{1. Betreuer: Prof. Dr. rer. nat Edgar Seemann\\
2. Betreuer: Dipl.-Ing. Florian Weißhardt}
\titlehead{\includegraphics[height=1.5cm]{images/hfu_logo.pdf}
\hfill
\includegraphics[height=1.5cm]{images/ipa_xl}}
\date{\today}
  \maketitle
  

\end{titlepage}
\restoregeometry


\section*{Kurzfassung}
\subsubsection*{Thema der Arbeit: Automatische Kalbrierung eines mobilen Serviceroboters}


\begin{tabular}{ll}

  Bearbeiter:&Jannik Sven Abbenseth\\\\
  1. Betreuer&Prof. Dr. rer. nat Edgar Seemann\\
  2. Betreuer&Dipl.-Ing. Florian Weißhardt \\\\
  Semester& Maschinenbau und Mechatronik, WS 2012/2013
\end{tabular}
\subsubsection{Kurzfassung}
\label{ssub:Kurzfassung}
Das Ziel dieser Arbeit ist es, die automatische Kalibrierung eines mobilen 
Serviceroboters anzupassen und zu generalisieren.
Hierdurch soll ein einheitliches Kalibrierverfahren 
geschaffen werden, mit dem alle am Institut für Produktionstechnik
und Automatisierung
eingesetzten Serviceroboter kalibriert werden können.

Dazu wird das bisher eingesetzte Verfahren analysiert und 
Verbesserungspotentiale festgestellt.

Die beim vorhandenen Verfahren notwendigen Datenaufnahmeschritte konnten nach Abschluss der 
Arbeit auf einen Schritt reduziert werden. Bisher wurden Denavit-Hartenberg 
Parameter verwendet. Diese wurden durch Transformationen ersetzt, dadurch ist die 
Kalibrierung an weiteren Robotern einsetzbar.

Zusätzlich wurde ein Verfahren entwickelt, mit dem ein Kalibrierobjekt durch 2D
Laserscanner erkannt werden kann, um diese in späteren Arbeiten zu kalibrieren.

\vfill
\textbf{Schlüsselwörter: }Robotik, Serviceroboter, Kalbrierung


\section*{Abstract\phantom{g}}
\subsubsection*{Title of Bachelor-Thesis: Automatic calibration of a mobile 
  service robot\phantom{g}}


\begin{tabular}{ll}

  Author:&Jannik Sven Abbenseth\\\\
  1. Examiner&Prof. Dr. rer. nat Edgar Seemann\\
  2. Examiner&Dipl.-Ing. Florian Weißhardt \\\\
  Semester& Mechanical Engineering and Mechatronics, WS 2012/2013
\end{tabular}
\subsubsection{Abstract}
\label{ssub:abstract}

This Bachelor-Thesis aims for adapting and generalizing the automatic 
calibration process of a mobile service robot.
Hereby a consistent calibration process for all used service robots at the
Fraunhofer Institute for Manufacturing Engineering and Automation should be
developed.

Therefore the former used calibration prozess was analyzed and its potential for
improvement determined.

During the implementation the two data collection steps for the camera and the
hand eye calibration were combined into a single step of data collection.
Another improvement was made by reorganizing the calculation step to compute 
the hand eye transformations with recorded transformations instead of Denavit-Hartenberg
parameters, which led to a wider range of application on service robots.

Additionally a detector for detecting a calibration pattern in 2D laser range finder
was implemented. It could be used in future works for calibrating the mount position
of these laser range finders.
\vfill
\textbf{Keywords: }Calibration, Robotics


\section*{Eidesstattliche Erklärung}
\label{sec:Eidesstattliche Erklärung}

\vskip5em
Ich erkläre hiermit an Eides statt, dass ich die vorliegende Arbeit selbstständig
und ohne unzulässige fremde Hilfe angefertigt habe.

Die verwendeten Literaturquellen sind im Literaturverzeichnis vollständig zitiert.


\vskip5em
Villingen-Schwenningen, \today 
\\
\vskip2em
\rule[-0.2cm]{5cm}{0.5pt}
  \\Jannik Sven Abbenseth

\tableofcontents

\listoffigures

%\listoftables


%
% EOF
%
