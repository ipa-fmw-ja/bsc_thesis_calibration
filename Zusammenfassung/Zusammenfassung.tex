%% Erläuterungen zu den Befehlen erfolgen unter
%% diesem Beispiel.
\documentclass{scrartcl}
 
\usepackage[utf8]{inputenc}
\usepackage[T1]{fontenc}
\usepackage{lmodern}
\usepackage[ngerman]{babel}
 
\title{Automatische Kalibrierung eines mobilen Serviceroboters}
\author{Jannik Abbenseth}
\date{WS 12/13}
\publishers{Betreuer: Prof. Dr. rer. nat Edgar Seemann}

\newcommand{\cob}{Care-O-bot}
\begin{document}
 
\maketitle
\section{Einleitung}

Der während der Bachelorthesis behandelte \cob\ ist ein Serviceroboter, der
in ihrer Mobilität eingeschränkten Menschen im Haushalt unterstützen soll. 
Dazu muss eine sichere Interaktion mit Menschen auch in wechselnden Umgebungen
möglich sein.

Der \cob besteht aus einer mobilen Plattform an der verschiedene Komponenten 
installiert sind. Dazu gehört ein Roboterarm sowie ein beweglicher Kopf mit 
Kameras. Auf dem \cob\ wird das Robot Operating System ROS eingesetzt. 

Um eine sichere Interaktion zu ermöglichen, muss ein exaktes Modell des 
Roboters vorhanden sein, mit dem Bewegungen berechnet werden können.
Dazu müssen sowohl die Kameras als auch die Montagepositionen aller an einer
Bewegung beteiligten Aktoren bekannt sein.

Zu Beginn der Arbeit liegt ein Kalibrierverfahren vor, mit dem ein \cob\ 
kalibriert werden konnte. Dieses Verfahren soll während der Bachelorthesis
angepasst werden um auch auf anderen Robotern eingesetzt werden zu können.


\section{Verbesserungspotential}
\label{sec:Verbesserungspotential}

Die Kalibrierung ist im Ausgangszustand durch drei Faktoren an den \cob3-3 
gebunden. Ein Faktor ist die feste Parametrisierung im Quellcode der
automatischen Kalibrierung. Zusätzlich werden Positionen an denen Daten für 
die Kalibrierung aufgenommen werden als Gelenkwinkel abgelegt. Durch 
unterschiedliche Aktoren an den Robotern führen diese Gelenkwinkel nicht immer
zu Positionen an denen ein gültiges Sample aufgenommen werden kann. Ein weiteres
Problem ist die Verwendung von Denavit-Hartenberg Parametern zur Berechnung der
Modelle der Aktoren. Diese Parameter waren nicht für alle verwendeten Aktoren
verfügbar.

\section{Umsetzung}
Zur Kalibrierung werden Änderungen am ROS-Stack \texttt{cob\_calibration}
vorgenommen. Um das Verfahren auch an Robotern einzusetzen, deren Aktoren nicht 
durch Denavit-Hartenberg Parameter beschrieben werden, muss die Berechnung der
Vorwärtskinematik geändert werden. Dazu wurden die vorhandenen Funktionen 
umgeschrieben um die schon vorhandene Roboterbeschreibung zu nutzen. Um dies
zu ermöglichen müssen Änderungen an der Datenaufnahme und den Schnittstellen
zwischen Datenaufnahme und Berechnung durchgeführt werden.

Die angegebenen Gelenkwinkel der Aufnahmepositionen werden durch eine Strategie
ersetzt, die neue Aufnahmepositionen für neue Roboter berechnet. Dafür werden
Positionen entlang eines Pfades mathematisch auf ihre Eignung als
Kalibrierungsposition Überprüft. Für die so berechneten Positionen werden 
anschließend die Gelenkwinkel berechnet und für die Datenaufnahme verwendet.

Als letzter Verbesserungsschritt werden alle Parameter die spezifisch für einen
Roboter sind ausgelagert und können in Konfigurationsdateien angepasst werden.

Durch die vorgenommenen Änderungen können einfach Konfigurationsdateien für neue
Roboter angelegt werden um diese zu Kalibrieren.


Außerdem wird ein verfahren vorgestellt, mit dem ein Kalibrierobjekt anhand
von 2D Laserscannerdaten erkannt werden kann. Dies dient zur Kalibrierung der
Montagepositionen der Laserscanner. In der bisherigen Kalibrierung werden nur
die Montagepositionen des Torsos, des Arms und der Kameras sowie die intrinsischen
Parameter der Kameras kalibriert.


\section{Fazit}

Das Hauptziel der Bachelorthesis eine automatische Kalibrierung für die
wichtige Kette von der Objekterkennung anhand der Kameras über den Torso und
den Arm zum Endeffektor zu entwickeln, die auf allen eingesetzten Robotern
funktioniert konnte erreicht werden.
Außerdem wurde durch die Entwicklung einer Objekterkennung anhand von 2D 
Laserscandaten die Aufnahme der Laserscanner in die Kalibrierung vorbereitet
und kann in folgenden Arbeiten umgesetzt werden.


\end{document}
